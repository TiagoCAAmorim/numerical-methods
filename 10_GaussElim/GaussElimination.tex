\documentclass[final,5p]{elsarticle}

% \documentclass[preprint,12pt]{elsarticle}

%% Use the option review to obtain double line spacing
%% \documentclass[authoryear,preprint,review,12pt]{elsarticle}

%% Use the options 1p,twocolumn; 3p; 3p,twocolumn; 5p; or 5p,twocolumn
%% for a journal layout:
% \documentclass[final,1p,times]{elsarticle}
%% \documentclass[final,1p,times,twocolumn]{elsarticle}
% \documentclass[final,3p,times]{elsarticle}
%% \documentclass[final,3p,times,twocolumn]{elsarticle}
% \documentclass[final,5p,times]{elsarticle}
%% \documentclass[final,5p,times,twocolumn]{elsarticle}
\usepackage[portuguese]{babel}

%% For including figures, graphicx.sty has been loaded in
%% elsarticle.cls. If you prefer to use the old commands
%% please give \usepackage{epsfig}

%% The amssymb package provides various useful mathematical symbols
\usepackage{amssymb}
\usepackage{amsmath}
\usepackage{multirow}

\usepackage{pgfplots}
\pgfplotsset{compat=1.18}
\usepgfplotslibrary{statistics}
\usepackage{pgfplotstable}

\usepackage{placeins}
\usepackage{hyperref}
\numberwithin{equation}{section}

\usepackage{algorithm}
\usepackage[noEnd=true, indLines=true]{algpseudocodex}
\algrenewcommand\algorithmicrequire{\textbf{Entrada:}}
\algrenewcommand\algorithmicwhile{\textbf{Enquanto}}
\algrenewcommand\algorithmicrepeat{\textbf{Repete}}
\algrenewcommand\algorithmicuntil{\textbf{Até}}
\algrenewcommand\algorithmicif{\textbf{Se}}
\algrenewcommand\algorithmicthen{\textbf{então}}
\algrenewcommand\algorithmicelse{\textbf{Caso contrário}}
\algrenewcommand\algorithmicensure{\textbf{Objetivo:}}
\algrenewcommand\algorithmicreturn{\textbf{Retorna:}}
\algrenewcommand\algorithmicdo{\textbf{faça}}
\algrenewcommand\algorithmicforall{\textbf{Para todos}}
\algnewcommand{\LineComment}[1]{\State \(\triangleright\) \textcolor{black!50}{\emph{#1}}}

% \usepackage[fleqn]{nccmath}
% \usepackage{multicol}


%=========== Gloabal Tikz settings
% \pgfplotsset{compat=newest}
% \usetikzlibrary{math}
% \pgfplotsset{
%     height = 10cm,
%     width = 10cm,
%     tick pos = left,
%     legend style={at={(0.98,0.30)}, anchor=east},
%     legend cell align=left,
%     }
%  \pgfkeys{
%     /pgf/number format/.cd,
%     fixed,
%     precision = 1,
%     set thousands separator = {}
% }

%% The amsthm package provides extended theorem environments
%% \usepackage{amsthm}

%% The lineno packages adds line numbers. Start line numbering with
%% \begin{linenumbers}, end it with \end{linenumbers}. Or switch it on
%% for the whole article with \linenumbers.
%% \usepackage{lineno}

\usepackage{listings}
\usepackage{xcolor}

\definecolor{codegreen}{rgb}{0,0.6,0}
\definecolor{codegray}{rgb}{0.5,0.5,0.5}
\definecolor{codepurple}{rgb}{0.58,0,0.82}
\definecolor{backcolour}{rgb}{0.98,0.98,0.98}

\lstdefinestyle{mystyle}{
    backgroundcolor=\color{backcolour},
    commentstyle=\color{codegreen},
    keywordstyle=\color{magenta},
    numberstyle=\tiny\color{codegray},
    stringstyle=\color{codepurple},
    basicstyle=\ttfamily\footnotesize,
    breakatwhitespace=false,
    breaklines=true,
    captionpos=b,
    keepspaces=true,
    numbers=left,
    numbersep=5pt,
    showspaces=false,
    showstringspaces=false,
    showtabs=false,
    tabsize=2
}

\lstset{style=mystyle}

% \journal{Nuclear Physics B}

\begin{document}

\begin{frontmatter}

%% Title, authors and addresses

%% use the tnoteref command within \title for footnotes;
%% use the tnotetext command for theassociated footnote;
%% use the fnref command within \author or \address for footnotes;
%% use the fntext command for theassociated footnote;
%% use the corref command within \author for corresponding author footnotes;
%% use the cortext command for theassociated footnote;
%% use the ead command for the email address,
%% and the form \ead[url] for the home page:
%% \title{Title\tnoteref{label1}}
%% \tnotetext[label1]{}
%% \author{Name\corref{cor1}\fnref{label2}}
%% \ead{email address}
%% \ead[url]{home page}
%% \fntext[label2]{}
%% \cortext[cor1]{}
%% \affiliation{organization={},
%%             addressline={},
%%             city={},
%%             postcode={},
%%             state={},
%%             country={}}
%% \fntext[label3]{}

\title{Performance do Método de Eliminação de Gauss para Resolver Sistema de Equações de Problemas de Fluxo em Meio Poroso\tnoteref{label_title}}
\tnotetext[label_title]{Relatório número 10 como parte dos requisitos da disciplina IM253: Métodos Numéricos para Fenômenos de Transporte.}

%% use optional labels to link authors explicitly to addresses:
%% \author[label1,label2]{}
%% \affiliation[label1]{organization={},
%%             addressline={},
%%             city={},
%%             postcode={},
%%             state={},
%%             country={}}
%%
%% \affiliation[label2]{organization={},
%%             addressline={},
%%             city={},
%%             postcode={},
%%             state={},
%%             country={}}

\author{Tiago C. A. Amorim\fnref{label_author}}
\tnotetext[label_author]{Atualmente cursando doutorado no Departamento de Engenharia de Petróleo da Faculdade de Engenharia Mecânica da UNICAMP (Campinas/SP, Brasil).}
\ead{t100675@dac.unicamp.br}
\affiliation[Tiago C. A. Amorim]{organization={Petrobras},%Department and Organization
addressline={Av. Henrique Valadares, 28},
city={Rio de Janeiro},
postcode={20231-030},
state={RJ},
country={Brasil}}

\begin{abstract}


\end{abstract}


%%Graphical abstract
% \begin{graphicalabstract}
%\includegraphics{grabs}
% \end{graphicalabstract}

%%Research highlights
% \begin{highlights}
% \item Research highlight 1
% \item Research highlight 2
% \end{highlights}

\begin{keyword}
    Método de Eliminação de Gauss \sep Fluxo em Meio Poroso
%% keywords here, in the form: keyword \sep keyword

%% PACS codes here, in the form: \PACS code \sep code

%% MSC codes here, in the form: \MSC code \sep code
%% or \MSC[2008] code \sep code (2000 is the default)

\end{keyword}

\end{frontmatter}

%% \linenumbers

%% main text
\section{Introdução}



\section{Metodologia}

    \subsection{Método de Eliminação de Gauss}


    Dado um sistema de $m$ equações diferenciais na forma:

    \begin{align}
        \frac{dy_1}{dt} &= f_1(t, \overrightarrow{y}) \nonumber \\
        \frac{dy_2}{dt} &= f_2(t, \overrightarrow{y}) \nonumber \\
        \vdots \nonumber \\
        \frac{dy_m}{dt} &= f_m(t, \overrightarrow{y}) \label{eq:pvi}
    \end{align}
    \noindent com
    \begin{align}
        \overrightarrow{y} = \left\{ y_1, y_2, \ldots, y_m \right\} \nonumber
    \end{align}
    \noindent e condições iniciais
    \begin{align}
        y_1(t_0) = \alpha_1, \; y_1(t_0) = \alpha_2, \; \ldots, \; y_m(t_0) = \alpha_m \nonumber
    \end{align}


    utilizado. Usualmente são apresentados de forma tabular:

    \bigskip
    \renewcommand{\arraystretch}{1.2}
    \begin{tabular}{c|ccccc}
        0         &                &                &          &                  &    \\
        $\beta_2$ & $\gamma_{2,1}$ &                &          &                  &    \\
        $\beta_3$ & $\gamma_{3,1}$ & $\gamma_{3,2}$ &          &                  &    \\
        $\vdots$  & $\vdots$       &                & $\ddots$ &                  &    \\
        $\beta_r$ & $\gamma_{r,1}$ & $\gamma_{r,2}$ & $\cdots$ & $\gamma_{r,r-1}$ &    \\
        \hline
        & $\lambda_1$    & $\lambda_2$    & $\cdots$ & $\lambda_{r-1}$  & $\lambda_r$ \\
    \end{tabular}
    \bigskip

    \subsection{Fluxo Bifásico Incompressível em Meio Poroso}

        As equações que regem o problema de fluxo em meio poroso deriva fundamentalmente da equação de conservação de massa \ref{eq:consmassa} com a equação de fluxo em meio poroso\ref{eq:darcy}, a lei de Darcy \cite{dake1983fundamentals}.

        \begin{align}
            &\sum_{p} \nabla \cdot  (y_{cp} \rho_p v_p) + \sum_{p} (y_{cp} \rho_p q_p) + \sum_{p} \frac{\partial}{\partial t} \left( \phi y_{cp} \rho_p S_p\right) = 0 \label{eq:consmassa} \\
            &v_p = - \frac{k k_{rp}}{\mu_p} \left( \frac{\partial p_p}{\partial x} - \gamma_p \frac{\partial D}{\partial x} \right) = - \frac{k k_{rp}}{\mu_p} \left( \frac{\partial \Phi_p}{\partial x} \right)\label{eq:darcy}
        \end{align}

        A maioria dos simuladores de fluxo comerciais utilizam diferenças finitas \cite{computer2022cmg}\cite{schlumberger2009technical}. Podemos acoplar as equações \ref{eq:consmassa} e \ref{eq:darcy}, e aplicar uma discretização no tempo ($\Delta t$) e no espaço ($\Delta x$) para um volume de controle ($V = \Delta x \Delta y \Delta z$). A forma unidimensional desta equação discretizada é apresentada a seguir \ref{eq:geralumd}.

        \begin{align}
            \sum_{p} & \left( \frac{\Delta y \Delta z}{\Delta x} y_{cp} \rho_p \frac{k k_{rp}}{\mu_p} \right)_{i+\tfrac{1}{2}} (\Phi_{p,i+1} - \Phi_{p,i})  \nonumber \\
            & + \left( \frac{\Delta y \Delta z}{\Delta x} y_{cp} \rho_p \frac{k k_{rp}}{\mu_p} \right)_{i-\tfrac{1}{2}} (\Phi_{p,i-1} - \Phi_{p,i}) \nonumber \\
            & + q_{cp}^{w} = \frac{1}{\Delta t} \sum_{p} (V \phi y_{cp} \rho_p S_p)_i^{t_i+\Delta t} - (V \phi y_{cp} \rho_p S_p)_i^{t_i} \label{eq:geralumd}
        \end{align}

        Na sua forma mais geral o problema de fluxo em meio poroso precisa ser resolvido para cada um dos componentes que constituem as fases envolvidas. É comum utilizar uma abordagem simplificada, em que poucos componentes são utilizados para representar os fluidos envolvidos. Esta simplificação é usualmente aplicada quando as trocas de fase são \emph{bem comportadas} e passíveis de serem representadas por um conjunto de tabelas, em substituição às equações de estado. Esta abordagem é conhecida como \emph{Black-Oil}. As equações também se simplificam ao substituir as densidades ($\rho_p$) e concentrações de componentes ($y_{cp}$) pelo fator volume de formação da fase ($B$) e as relações volumétricas entre as fases e os componentes ($R$)\footnote{Ver dedução completa em \cite{dake1983fundamentals}.}.

        Quando apenas água e óleo estão envolvidos na simulação, apenas duas equações de conservação de massa são necessárias. Como a soma das saturações é igual à unidade ($S_w + S_o = 1$) e negligenciando a tensão interfacial entre a água e o óleo ($p_o = p_w = p$), para cada volume de controle é preciso resolver apenas duas variáveis: $p$ e $S_w$.

        \begin{align}
            \left( \frac{\Delta y \Delta z}{\Delta x} \lambda_w \right)&_{i+\tfrac{1}{2}} (p_{w,i+1} - p_{w,i} - \gamma_w \Delta D_{i+\tfrac{1}{2}})  \nonumber \\
            + \left( \frac{\Delta y \Delta z}{\Delta x} \lambda_w \right)&_{i-\tfrac{1}{2}} (\Phi_{w,i-1} - \Phi_{w,i}) \nonumber \\
            &  = \frac{1}{\Delta t} \left(\frac{V \phi S_w}{B_w}\right)_i^{t_i+\Delta t} - \left(\frac{V \phi S_w}{B_w}\right)_i^{t_i} + q^{std}_w\label{eq:blackoilumd}
        \end{align}

        \noindent com:
        \begin{align}
            \lambda_p = \frac{k k_{rp}}{B_p \mu_p} \nonumber
        \end{align}





    \subsection{Modelo de Reservatório Proposto}

        necessárias apenas duas equações diferenciais. Para um problema em que a vazão de óleo é função da pressão no reservatório, duas novas equações diferenciais são adicionadas, governando $\frac{dNp}{dt}$ e $Np$.

\section{Implementação} \label{sec:implementacao}

        Todo o código utilizado nesta análise foi desenvolvido em C++. Foram criados objetos próprios para cada elemento integrante do problema proposto:

        \begin{description}
            \item[IVPSystem] Classe que define um problema de valor inicial na forma.
            \begin{itemize}
                \item O usuário precisa especificar $f_i(t,\overrightarrow{y})$, $a$ (tempo inicial), $b$ (tempo final), $n$ (número de passos de tempo) e $y_i(a)$ (valores iniciais).
                \item O usuário também pode especificar as soluções exatas ($y_i=f_i(t)$), para calcular o erro de aproximação.
            \end{itemize}
            \item[Fetkovich] Classe que resolve o comportamento de um aquífero como proposto por Fetkovich.
            \begin{itemize}
                \item O usuário precisa definir as características do aquífero e prover uma função que retorne a pressão na interface do aquífero com o reservatório. Esta função depende do tempo e do influxo acumulado de água do aquífero para o reservatório ($W_e$).
                \item Uma modificação foi feita na Implementação do Método de Fetkovich com relação aos testes apresentados nos relatórios anteriores.
                \begin{itemize}
                    \item Como existe um termo implícito a ser resolvido no método, anteriormente foi admitido realizar um cálculo com até 20 iterações.
                    \item Aplicando a mesma filosofia dos métodos preditor-corretor, este limite agora é de apenas uma iteração. E nos testes realizados os resultados foram muito parecidos.
                    \item Desta forma o Método de Fetkovich agora tem apenas duas avaliações da função de pressão a cada passo de tempo.
                \end{itemize}
            \end{itemize}
        \end{description}

\section{Resultados}

        de $y(t)$ incluída nos cálculos de Runge-Kutta tem um comportamento mais suave, sem as oscilações da integração numérico com Simpson e Trapézio composto.

        % \begin{figure}[hbt!]
        %     \begin{tikzpicture}
        %         \begin{semilogyaxis}[
        %             grid=both,
        %             xlabel = {$t$},
        %             ylabel = {$|w_i-y(t)|/|y(t)|$},
        %             legend style={at={(0.90,0.85)}, anchor=east, font=\footnotesize},
        %             ]
        %             \addplot[color=blue, solid, smooth, thick] table [x=t, y=S_error] {test1_rungekutta4_intN.txt};
        %             \addplot[color=black, solid, smooth, thick] table [x=t, y=y2_err] {test1_rungekutta4_intRK.txt};
        %             \legend{Simpson+Trapézio, Runge-Kutta};
        %         \end{semilogyaxis}
        %     \end{tikzpicture}
        %     \caption{Erro da aproximação da acumulada de $y(t)$ com o método de Runge-Kutta de quarta ordem e por integração numérica com Simpson e Trapézio Composto, do PVI número \ref{item:pvi1}.}
        %     \label{fig:teste1}
        % \end{figure}


        O código foi implementado em C++ e em um único arquivo. Pode ser encontrado em \href{https://github.com/TiagoCAAmorim/numerical-methods/blob/main/10_GaussElim/10_GaussElim.cpp}{https://github.com/Tiago CAAmorim/numerical-methods}.

    \section{Conclusão}

        Foi possível testar problemas mais desafiadores ao passar de métodos de aproximação numérica de uma equação diferencial de primeira ordem para métodos que resolvem um sistema de equações. Para o problema proposto o Método de Runge-Kutta teve desempenho pior que o do Método de Fetkovich.

    % \label{}

%% The Appendices part is started with the command \appendix;
%% appendix sections are then done as normal sections

\appendix

\section{Lista de Variáveis}

\begin{description}
    \item[$Bo$:]Fator volume de formação do óleo no reservatório ($m^3/m^3$).
    \item[$Bw$:]Fator volume de formação da água no reservatório ($m^3/m^3$).
    \item[$\Delta x$:]Discretização espacial na direção $i$.
    \item[$\Delta y$:]Discretização espacial na direção $j$.
    \item[$\Delta z$:]Discretização espacial na direção $k$.
    \item[$\Delta t$:]Discretização temporal.
    \item[$\gamma_p$:]Peso específico da fase $p$ ($\gamma_p = \rho_p g$).
    \item[$k$:]Permeabilidade absoluta do meio poroso.
    \item[$k_{rp}$:]Permeabilidade relativa da fase $p$.
    \item[$\mu_p$:]Viscosidade da fase $p$.
    \item[$p_p$:]Pressão da fase $p$.
    \item[$\phi$:]Porosidade da rocha.
    \item[$q_p$:]Vazão volumétrica da fase $p$.
    \item[$q^w_{cp}$:]Vazão mássica do componente $c$ na fase $p$.
    \item[$\rho_p$:]Densidade da fase $p$.
    \item[$S_p$:]Saturação da fase $p$ no meio poroso.
    \item[$V$:]Volume total do volume de controle.
    \item[$v_p$:]Velocidade da fase $p$.
    \item[$y_{cp}$:]Concentração do componente $c$ na fase $p$.


    % \item[$c_r$:]Compressibilidade do volume poroso ($1/bar$).
    % \item[$c_{o,b}$:]Compressibilidade do óleo na pressão de bolha ($1/bar$).
    % \item[$c_{aq}$:]Compressibilidade total do aquífero ($1/bar$).
    % \item[$J$:]Índice de produtividade do aquífero ($m^3/d/bar$).
    % \item[$N$:]Volume de óleo no reservatório, medido em condições padrão ($m^3$).
    % \item[$N_i$:]Volume de óleo no reservatório inicial, medido em condições padrão ($m^3$).
    % \item[$Np$:]Volume de óleo produzido, medido em condições padrão ($m^3$).
    % \item[$p_{aq}$:]Pressão média do aquífero ($bar$).
    % \item[$p_{i,aq}$:]Pressão inicial do aquífero ($bar$).
    % \item[$p_b$:]Pressão de bolha do óleo ($bar$).
    % \item[$p_{res}$:]Pressão na interface entre o aquífero e o reservatório ($bar$).
    % \item[$p_{i,res}$:]Pressão inicial na interface entre o aquífero e o reservatório ($bar$).
    % \item[$Pv$:]Volume poroso no reservatório ($m^3$).
    % \item[$Pv_i$:]Volume poroso no reservatório na pressão inicial ($m^3$).
    % \item[$Q_o$ ou $\frac{dNp}{dt}$:]Vazão de óleo produzido ($m^3/d$).
    % \item[$Q_w$ ou $\frac{dW_e}{dt}$:]Vazão de água do aquífero para o reservatório ($m^3/d$).
    % \item[$\Delta t_j$:]Diferença entre os tempos $t_{j-1}$ e $t_j$ ($d$).
    % \item[$W_e$:]Volume de água acumulado do aquífero para o reservatório ($m^3$).
    % \item[$W_{e,max}$:]Máximo influxo de água \emph{possível} do aquífero para o reservatório ($m^3$).
    % \item[$(\Delta W_e)_j$:]Influxo de água entre os tempos $t_{j-1}$ e $t_j$ ($m^3$).
    % \item[$W_{i,aq}$:]Volume inicial do aquífero ($m^3$).
    % \item[$W_{res}$:]Volume de água no reservatório ($m^3$).
    % \item[$W_{i,res}$:]Volume inicial de água no reservatório ($m^3$).
\end{description}

%% \section{}
%% \label{}

%% If you have bibdatabase file and want bibtex to generate the
%% bibitems, please use
%%

\bibliographystyle{elsarticle-num}
\bibliography{refs}

%% else use the following coding to input the bibitems directly in the
%% TeX file.

% \begin{thebibliography}{00}

%% \bibitem{label}
%% Text of bibliographic item

% \bibitem{}

% \end{thebibliography}

% \newpage
% \FloatBarrier
% \section{Código em C}

% O código de ambos métodos foi implementado em um único arquivo. O código é apresentado em duas partes neste documento para facilitar a leitura. O código pode ser encontrado em \href{https://github.com/TiagoCAAmorim/numerical-methods}{https://github.com/TiagoCAAmorim/numerical-methods}.

% \subsection{Método da Bissecção}
% \lstinputlisting[language=C, linerange={1-229}]{./02_newton_raphson.c}

% \subsection{Método de Newton-Raphson}
% \lstinputlisting[language=C, linerange={231-445}]{./02_newton_raphson.c}

% \subsection{Método da Mínima Curvatura}
% \lstinputlisting[language=C, linerange={448-958}]{./02_newton_raphson.c}

\end{document}
\endinput
\documentclass[final,5p]{elsarticle}

% \documentclass[preprint,12pt]{elsarticle}

%% Use the option review to obtain double line spacing
%% \documentclass[authoryear,preprint,review,12pt]{elsarticle}

%% Use the options 1p,twocolumn; 3p; 3p,twocolumn; 5p; or 5p,twocolumn
%% for a journal layout:
% \documentclass[final,1p,times]{elsarticle}
%% \documentclass[final,1p,times,twocolumn]{elsarticle}
% \documentclass[final,3p,times]{elsarticle}
%% \documentclass[final,3p,times,twocolumn]{elsarticle}
% \documentclass[final,5p,times]{elsarticle}
%% \documentclass[final,5p,times,twocolumn]{elsarticle}
\usepackage[portuguese]{babel}

%% For including figures, graphicx.sty has been loaded in
%% elsarticle.cls. If you prefer to use the old commands
%% please give \usepackage{epsfig}

%% The amssymb package provides various useful mathematical symbols
\usepackage{amssymb}
\usepackage{amsmath}
\usepackage{multirow}

\usepackage{pgfplots}
\pgfplotsset{compat=1.18}
\usepackage{placeins}
\usepackage{hyperref}
\numberwithin{equation}{section}

\usepackage{algorithm}
\usepackage[noEnd=true, indLines=true]{algpseudocodex}
\algrenewcommand\algorithmicrequire{\textbf{Entrada:}}
\algrenewcommand\algorithmicwhile{\textbf{Enquanto}}
\algrenewcommand\algorithmicrepeat{\textbf{Repete}}
\algrenewcommand\algorithmicuntil{\textbf{Até}}
\algrenewcommand\algorithmicif{\textbf{Se}}
\algrenewcommand\algorithmicthen{\textbf{então}}
\algrenewcommand\algorithmicelse{\textbf{Caso contrário}}
\algrenewcommand\algorithmicensure{\textbf{Objetivo:}}
\algrenewcommand\algorithmicreturn{\textbf{Retorna:}}
\algrenewcommand\algorithmicdo{\textbf{faça}}
\algnewcommand{\LineComment}[1]{\State \(\triangleright\) \textcolor{black!50}{\emph{#1}}}

% \usepackage[fleqn]{nccmath}
% \usepackage{multicol}


%=========== Gloabal Tikz settings
% \pgfplotsset{compat=newest}
% \usetikzlibrary{math}
% \pgfplotsset{
%     height = 10cm,
%     width = 10cm,
%     tick pos = left,
%     legend style={at={(0.98,0.30)}, anchor=east},
%     legend cell align=left,
%     }
%  \pgfkeys{
%     /pgf/number format/.cd,
%     fixed,
%     precision = 1,
%     set thousands separator = {}
% }

%% The amsthm package provides extended theorem environments
%% \usepackage{amsthm}

%% The lineno packages adds line numbers. Start line numbering with
%% \begin{linenumbers}, end it with \end{linenumbers}. Or switch it on
%% for the whole article with \linenumbers.
%% \usepackage{lineno}

\usepackage{listings}
\usepackage{xcolor}

\definecolor{codegreen}{rgb}{0,0.6,0}
\definecolor{codegray}{rgb}{0.5,0.5,0.5}
\definecolor{codepurple}{rgb}{0.58,0,0.82}
\definecolor{backcolour}{rgb}{0.98,0.98,0.98}

\lstdefinestyle{mystyle}{
    backgroundcolor=\color{backcolour},
    commentstyle=\color{codegreen},
    keywordstyle=\color{magenta},
    numberstyle=\tiny\color{codegray},
    stringstyle=\color{codepurple},
    basicstyle=\ttfamily\footnotesize,
    breakatwhitespace=false,
    breaklines=true,
    captionpos=b,
    keepspaces=true,
    numbers=left,
    numbersep=5pt,
    showspaces=false,
    showstringspaces=false,
    showtabs=false,
    tabsize=2
}

\lstset{style=mystyle}

% \journal{Nuclear Physics B}

\begin{document}

\begin{frontmatter}

%% Title, authors and addresses

%% use the tnoteref command within \title for footnotes;
%% use the tnotetext command for theassociated footnote;
%% use the fnref command within \author or \address for footnotes;
%% use the fntext command for theassociated footnote;
%% use the corref command within \author for corresponding author footnotes;
%% use the cortext command for theassociated footnote;
%% use the ead command for the email address,
%% and the form \ead[url] for the home page:
%% \title{Title\tnoteref{label1}}
%% \tnotetext[label1]{}
%% \author{Name\corref{cor1}\fnref{label2}}
%% \ead{email address}
%% \ead[url]{home page}
%% \fntext[label2]{}
%% \cortext[cor1]{}
%% \affiliation{organization={},
%%             addressline={},
%%             city={},
%%             postcode={},
%%             state={},
%%             country={}}
%% \fntext[label3]{}

\title{Aplicação do Método da Secante para Resolver o Cálculo Inverso do Método da Mínima Curvatura\tnoteref{label_title}}
\tnotetext[label_title]{Relatório número 3 como parte dos requisitos da disciplina IM253: Métodos Numéricos para Fenômenos de Transporte.}

%% use optional labels to link authors explicitly to addresses:
%% \author[label1,label2]{}
%% \affiliation[label1]{organization={},
%%             addressline={},
%%             city={},
%%             postcode={},
%%             state={},
%%             country={}}
%%
%% \affiliation[label2]{organization={},
%%             addressline={},
%%             city={},
%%             postcode={},
%%             state={},
%%             country={}}

\author{Tiago C. A. Amorim\fnref{label_author}}
\tnotetext[label_author]{Atualmente cursando doutorado no Departamento de Engenharia de Petróleo da Faculdade de Engenharia Mecânica da UNICAMP (Campinas/SP, Brasil).}
\ead{t100675@dac.unicamp.br}
\affiliation[Tiago C. A. Amorim]{organization={Petrobras},%Department and Organization
addressline={Av. Henrique Valadares, 28},
city={Rio de Janeiro},
postcode={20231-030},
state={RJ},
country={Brasil}}

\begin{abstract}

    A partir do Método da Mínima Curvatura, para calcular os parâmetros de perfuração de um poço direcional em função das coordenadas cartesianas da sua trajetória é preciso resolver uma equação implícita do tipo $g(x)=x$. Este tipo de problema pode ser resolvido com algoritmos que buscam a raiz de uma função de uma variável. O Método da Secante se mostrou mais eficiente que o Método de Newton-Raphson para resolver este problema, com convergência mais que quadrátrica.

\end{abstract}


%%Graphical abstract
% \begin{graphicalabstract}
%\includegraphics{grabs}
% \end{graphicalabstract}

%%Research highlights
% \begin{highlights}
% \item Research highlight 1
% \item Research highlight 2
% \end{highlights}

\begin{keyword}
    Método da Mínima Curvatura \sep Método da Secante
%% keywords here, in the form: keyword \sep keyword

%% PACS codes here, in the form: \PACS code \sep code

%% MSC codes here, in the form: \MSC code \sep code
%% or \MSC[2008] code \sep code (2000 is the default)

\end{keyword}

\end{frontmatter}

%% \linenumbers

%% main text
\section{Introdução}

Foi desenvolvido um método de cálculo de parâmetros de perfuração ($\Delta$S, $\theta$, $\phi$) em função de coordenadas cartesianas ($\Delta$N, $\Delta$E, $\Delta$V) da trajetória de um poço a partir da formulação do Método da Mínima Curvatura \cite{relatoriobisseccao}. A formulação desenvolvida é implícita, e tem a forma $g(x)=x$.

A primeira tentativa de resolver o problema proposto, neste relatório denominado de \textbf{Inverso da Mínima Curvatura}, foi com o Método da Bissecção. O Método da Bissecção conseguiu encontrar a resposta correta em todos os testes realizados. O segundo teste foi feito com o Método de Newton-Raphson que apresentou convergência mais que quadrática \cite{relatorionewtonraphson}.

Neste terceiro relatório é aplicado o Método da Secante para resolver o mesmo problema.

\section{Metodologia}

\subsection{Inverso da Mínima Curvatura}

O algoritmo de cálculo do inverso da mínima curvatura é discutido em \cite{relatoriobisseccao}. O algoritmo assume que para calcular os parâmetros de perfuração entre dois pontos quaisquer são conhecidos os parâmetros de perfuração do ponto inicial ($\theta_1$, $\phi_1$) e as distâncias cartesianas entre os pontos ($\Delta N$,$\Delta E$, $\Delta V$). O objetivo é calcular $\theta_2$, $\phi_2$ e $\Delta S$. As fórmulas desenvolvidas são repetidas neste relatório para dar mais clareza.

\begin{subequations}\label{eq:inversomcm}
    \begin{align}
        \cos \theta_2 =& 2 \frac{\Delta V}{\Delta S f(\alpha)} - \cos \theta_1 \label{eq:costheta2} \\
        \sin \theta_2 =& \sqrt{1 - \cos^2 \theta_2} \label{eq:sintheta2} \\
        \sin \phi_2 =& \frac{-\Delta N}{\Delta H^2} \left( \frac{\Delta \Psi \sin \theta_1}{\sin \theta_2} \right) \nonumber \\
        & + \frac{\Delta E}{\Delta H^2} \sqrt{\Delta H^2 - \left( \frac{\Delta \Psi \sin \theta_1}{\sin \theta_2} \right)^2} \label{eq:sinphi2} \\
        \cos \phi_2 =& \frac{\Delta E}{\Delta H^2} \left( \frac{\Delta \Psi \sin \theta_1}{\sin \theta_2} \right) \nonumber \\
        & + \frac{\Delta N}{\Delta H^2} \sqrt{\frac{1}{\Delta H^2} - \left( \frac{\Delta \Psi \sin \theta_1}{\sin \theta_2} \right)^2} \label{eq:cosphi2} \\
        \Delta S f(\alpha) =& 2 \sqrt{\frac{\Delta N^2 + \Delta E^2 + \Delta V^2}{A^2+B^2+C^2}} \label{eq:DeltaSfa}
    \end{align}
\end{subequations}

onde
\begin{align*}
    \Delta \Psi &= \Delta N \sin \phi_1 - \Delta E \cos \phi_1 \\
    \Delta H^2 &= \Delta N^2 + \Delta E^2 \\
    A &= \sin \theta_1 \cos \phi_1 + \sin \theta_2 \cos \phi_2 \\
    B &= \sin \theta_1 \sin \phi_1 + \sin \theta_2 \sin \phi_2 \\
    C &= \cos \theta_1 + \cos \theta_2
\end{align*}

Em \cite{relatoriobisseccao} são discutidos detalhes adicionais sobre a implementação deste algoritmo e seus limites. Em \cite{relatorionewtonraphson} é apresentada a derivação do problema em função da variável principal: $\Delta S f(\alpha)$.

    \subsection{Método da Secante}

    O código desenvolvido foi baseado no pseudo-código de \cite{burden2016analise}. Uma forma simplificada é apresentada no Algoritmo \ref{alg:secante}. O Método da Secante deriva do Método de Newton-Raphson, e faz uma aproximação da derivada da função de interesse: $f^{\prime}(x) \approx \Delta f(x) / \Delta x$. O método precisa de dois pontos iniciais, e, assim como o Método de Newton-Raphson, converge mais rápido que o Método da Bissecção, e não tem garantia de encontrar uma raiz.

    \begin{algorithm}
        \caption{Método da Secante}\label{alg:secante}
        \begin{algorithmic}
            \Require $x_0$, $x_1$
            \If{$|f(x_0)| < |f(x_1)|$}
                \State $x_{melhor} \gets x_0$
            \Else
                \State $x_{melhor} \gets x_1$
            \EndIf
            \State $i \gets 2$
            \Repeat
                \State $x_{i} \gets x_{i-1} - \frac{f(x_{i-1})}{f(x_{i-1})-f(x_{i-2})}(x_{i-1}-x_{i-2})$
                \If{$|f(x_{i+1})| < |f(x_{melhor})|$}
                    \State $x_{melhor} \gets x_{i+1}$
                \EndIf
                \State $i \gets i+1$
            \Until{$|x_i-x_{i-1}|<\epsilon$}
            \State \Return $x_{melhor}$
        \end{algorithmic}
    \end{algorithm}

    Foram adicionados critérios adicionais ao algoritmo para controlar o processo iterativo:

    \begin{itemize}
        \item Foram implementados dois métodos de cálculo do critério de convergência\footnote{Maiores detalhes em \cite{relatoriobisseccao}.}: \emph{Direto} e \emph{Relativo}.
        \item É feito um término prematuro do processo iterativo caso algum $|f(x)| < \epsilon$. O valor padrão de $\epsilon$ é $10^{-7}$ (variável \verb|epsilon| no código). Este teste também é feito antes de entrar no \emph{loop}.
        \item A cada iteração é verificado se a função está variando muito pouco: $|f(x_i)-f(x_{i-1})| < \epsilon$. Em caso positivo é gerada uma mensagem e é encerrado o \emph{loop}.
        \item Como o método não garante convergência, ao longo das iterações é guardado o melhor resultado ($x$ tal que $|f(x)|$ seja mínimo). Este é o resultado que é retornado.
    \end{itemize}

    \section{Resultados}

    Para facilitar a análise da qualidade do código desenvolvido, foram criadas funções que realizam diversos testes onde a resposta exata é conhecida:

    \begin{description}
        \item[tests\textunderscore secant()] Testa o Método da Secante em diferentes funções: linear, quadrática, exponencial e trigonométrica. Também foram aplicados casos específicos para o algoritmo tratar: raiz no ponto inicial, ponto inicial muito próximo da raiz, ponto inicial em um ponto de mínimo da função (derivada nula), uso de critério de convergência relativo, função com derivada nula na raiz, função com vários mínimos e máximos locais.

        \item[tests\textunderscore minimum\textunderscore curvature()] Testa o algoritmo proposto para resolver o inversa da mínima curvatura para diferentes geometrias de trecho de poço: poço vertical, horizontal, inclinado, em curva.
    \end{description}

    A implementação do Método da Secante foi adaptada, impondo limites máximo e mínimo para os pontos a serem testados. No problema do inverso da mínima curvatura a variável principal ($\Delta S f(\alpha)$) não pode ter um valor qualquer. Optou-se por não parar prematuramente o processo iterativo quando os limites são ultrapassados. Quando o novo valor da variável principal ultrapassa algum dos limites pré-estabelecidos, é utilizado o valor do limite que foi violado. Para o problema do inverso da mínima curvatura foi utilizado um limite inferior de $\Delta S f(\alpha)$ igual ao discutido em \cite{relatoriobisseccao}.

    Como esperado, as funções lineares convergiram em uma iteração, pois a aproximação da derivada feita no Método da Secante é exata neste exemplo. As demais funções polinomiais testadas convergiram rapidamente, mas com mais iterações que o Método de Newton Raphson (Tabela \ref{table:iteracoes}). O Método da Secante teve problemas de convergência com a função exponencial: $e^{x^2-4}-1$. O método só conseguiu convergir quando foram definidos valores iniciais próximos da raiz. Como esta função \emph{cresce} muito rapidamente em $x<-2$ e $x>2$, o termo $(f(x_{i-1})-f(x_{i-2}))$ sofre com problemas numéricos, pois é a subtração de números com grandezas muito distintas.

    Para as diferentes funções testadas, o Método da Bisseção convergiu linearmente em uma razão de aproximadamente 2 \cite{relatoriobisseccao}, e o Método de Newton-Raphson em geral teve convergência quadrática \cite{relatorionewtonraphson}. Com o Método da Secante a convergência foi aproximadamente quadrática, em geral um pouco pior que o Método de Newton-Raphson (Figura \ref{fig:erro}).

    \begin{table*}
        \caption{Comparação entre o número de iterações necessários para um mesmo critério de convergência usando os Métodos da Bissecção, Newton-Raphson e Secante.}
        \label{table:iteracoes}
        \begin{tabular}{ l c c c c c c c c c}
            \hline
            \multirow{2}{*}{Função} & \multirow{2}{*}{Raiz} & \multicolumn{3}{c}{Bissecção} & \multicolumn{2}{c}{Newton-Raphson} & \multicolumn{3}{c}{Secante}\\
            &  & $x_a$ & $x_b$ & Iter. & $x_0$ & Iter. & $x_0$ & $x_1$ & Iter. \\
            \hline
            Linear & 0.3 & 0. & 2. &  11 & 0. & 1 & 0 & 1 & 1\\
            Quadrática & -0.1 & -0.25 & 1. & 11 & 0.25 & 4 & 0.25 & 1. & 6 \\
            Exponencial  & 2. & 0. & 10. & 14 & 5. & 25 & 2.5 & 3. & 9 \\
            Trigonométrica & $3\pi/4$ & 0. & 5. & 13 & 3. & 3 & 0 & 3 & 4 \\
            1/4 círculo horizontal & 20 & 14.1421 & 120.417 & 17 & 14.4743 & 1 & 21.2132 & 15.5563 & 1 \\
            Seção 2 do poço em S & 131.652 & 130.526 & 1111.4 & 20 & 133.592 & 11 & 195.789 & 143.579 & 4 \\
            Poço 3D & 10.3144 & 9.84918 & 83.8634 & 17 & 10.0805 & 4 & 14.7738 & 10.8341 & 3 \\
            \hline
        \end{tabular}
    \end{table*}

    \begin{figure*}
        \begin{tikzpicture}
            \begin{axis}[
                height = 8cm, % standard 10cm
                width = 0.8\textwidth, %15cm,  % standard 10cm
                % xmode=log,
                ymode=log,
                grid=both,
                ymin=1e-6,
                ymax=1e1,
                xlabel = {Iterações},
                ylabel = {$\varepsilon_i = |P_i - P|$},
                legend style={at={(0.99,0.85)}, anchor=east},
                ]

                \addplot[color=black, solid, mark=*] table [x=Iteration, y =Linear] {IterationError.txt};
                \addplot[color=blue, solid, mark=*] table [x=Iteration, y =Quadratic] {IterationError.txt};
                \addplot[color=green, solid, mark=*] table [x=Iteration, y =Exponential] {IterationError.txt};
                \addplot[color=red, solid, mark=*] table [x=Iteration, y =Trigonometric] {IterationError.txt};
                \addplot[color=purple, solid, mark=*] table [x=Iteration, y =3Dwell] {IterationError.txt};
                \addplot[color=black, dashed] table [x=Iteration, y =Linear2] {IterationError.txt};
                \addplot[color=black, dashed] table [x=Iteration, y =Quad2] {IterationError.txt};
                \legend{Linear, Quadrática, Exponencial, Trigonométrica, Poço 3D};
            \end{axis}
        \end{tikzpicture}
        \caption{Evolução do erro do Método da Secante com diferentes funções (linhas tracejadas representam $\varepsilon_{i+1} / \varepsilon_i = 0.5$ e $\varepsilon_{i+1} / \varepsilon_i^2 = 0.5$).}
        \label{fig:erro}
    \end{figure*}

    O Método de Newton-Raphson se mostrou instável para algumas das funções de teste \cite{relatorionewtonraphson}, ficando muito dependente do ponto inicial para conseguir convergir. Uma função especialmente complicada para o Método de Newton-Raphson foi\footnote{Esta função é diferente da função 'Trigonométrica' listada na Tabela \ref{table:iteracoes} e na Figura \ref{fig:erro}, que é $\cos x + \sin x$.}: $3x + x^2 \sin^2 \frac{x\pi}{10}$.

    Com o Método da Secante não foi observado este problema, e diferentes valores iniciais conseguiram convergir para a raiz do problema. Os valores de $x_i$ \textbf{não} convergiram monotonicamente para a raiz, mas conseguiram encontrar o resultado correto (Figura \ref{fig:ftrigonometrica}). Aparentemente, ao avaliar dois pontos em uma região com derivada perto de zero, o método irá \emph{pular} para um novo ponto longe do anterior, e na próxima iteração serão utilizados dois pontos distantes entre si. Ao fazer a inclinação média entre os dois pontos distantes o método irá dar maior relevância à tendência mais geral da função, e menor às variações de maior frequência.

    O Método da Secante precisa de dois valores iniciais de $x$, e a ordem em que estes valores são fornecidos pelo usuário pode impactar o resultado. A Figura \ref{fig:erroinv} mostra que, para funções \emph{bem comportadas}, é uma boa prática entrar um valor de $x_1$ mais próximo da raiz que o valor de $x_0$. Ao fazer isto, o valor de $x_1$ é que será utilizado na 2$^a$ iteração. Para as funções testadas o impacto foi pequeno, mas foi positivo seguir a boa prática. O impacto foi maior apenas na avaliação da função trigonométrica (Figura \ref{fig:ftrigonometrica}). Neste exemplo a convergência foi muito impactada e a raiz encontrada pelos dois testes foi diferente.

    \begin{figure*}
        \centering
        \begin{tikzpicture}
            \begin{axis}[
                height = 8cm, % standard 10cm
                width = 0.8\textwidth, %15cm,  % standard 10cm
                % xmode=log,
                ymode=log,
                grid=both,
                ymin=1e-6,
                ymax=1e3,
                xlabel = {Iterações},
                ylabel = {$\varepsilon_i = |P_i - P|$},
                legend style={at={(0.99,0.85)}, anchor=east},
                ]

                \addplot[color=black, solid, mark=*] table [x=Iteration, y =Normal] {InvertedPoints_3rdDegree.txt};
                \addplot[color=blue, solid, mark=*] table [x=Iteration, y =Normal] {InvertedPoints_Exp.txt};
                \addplot[color=red, solid, mark=*] table [x=Iteration, y =Normal] {InvertedPoints_Trig.txt};

                \addplot[color=black, dashed, mark=*] table [x=Iteration, y =Inv] {InvertedPoints_3rdDegree.txt};
                \addplot[color=blue, dashed, mark=*] table [x=Iteration, y =Inv] {InvertedPoints_Exp.txt};
                \addplot[color=red, dashed, mark=*] table [x=Iteration, y =Inv] {InvertedPoints_TrigInv.txt};

                \legend{Polinômio 3$^o$ grau, Exponencial, Trigonométrica};
            \end{axis}

        \end{tikzpicture}
        \caption{Sensibilidade da evolução do erro do Método da Secante em função dos pontos iniciais: $x_0$ e $x_1$ (Linhas contínuas tem $|f(x_1)|<|f(x_0)|$, e as tracejadas o inverso).}
        \label{fig:erroinv}
    \end{figure*}

    Na aplicação do Método da Secante para resolver o inverso da mínima curvatura os resultados foram ainda melhores que os do Método de Newton Raphson, que já foram muito bons. O lado \emph{à direita} da raiz tem forma próxima de uma reta (Figura \ref{fig:fpoco3d}), e a aproximação da derivada do Método da Secante por uma inclinação \emph{média} se mostrou muito eficaz logo na primeira iteração. Nos testes realizados esta primeira iteração, a partir dos pontos iniciais fornecidos pelo usuário, conseguiram uma aproximação muito boa da raiz.
    No código foram utilizados por padrão valores 50\% e 10\% maior que o mínimo como $x_0$ e $x_1$, respectivamente.

    O código dos três métodos foi implementado em um único arquivo, e pode ser encontrado em \href{https://github.com/TiagoCAAmorim/numerical-methods/blob/main/03_Secant/03_secant.c}{https://github.com/ TiagoCAAmorim/numerical-methods}.

    \section{Conclusão}

    O Método da Secante apresentou desempenho melhor que o do Método de Newton-Raphson para fazer o cálculo de parâmetros de perfuração em função de coordenadas cartesianas. O formato da função de interesse para resolver o problema proposto levou a uma estimativa muito boa da solução na primeira iteração em todos os testes realizados. Além de alcançar convergência em um número menor de iterações, o Método da Secante tem a vantagem adicional de não precisar calcular a derivada da função, ou seja, reduzindo ainda mais o custo computacional necessário para resolver o problema proposto.

    % \label{}

%% The Appendices part is started with the command \appendix;
%% appendix sections are then done as normal sections

\appendix

\section{Funções de Teste}

Nos gráficos a seguir as curvas pretas são as funções de teste. A raiz (ou raízes) está denotada pelo quadrado azul. São apresentadas apenas algumas das funções de teste. Em vermelho estão os pontos avaliados ao longo de um dos testes realizados.

\begin{figure}[hbt!]
    \begin{tikzpicture}
        \begin{axis} [axis lines=center]
            \addplot [domain=-3:3, smooth, thick] { 3*x*x*(x + 2) };
            % \addplot [dashed, domain=-3:3, smooth, thick] { 3*x*(3*x + 4) };
            \addplot[color=red, mark=o, mark size=2.pt] coordinates {
                (   2.69717,         102.512)
                (   1.93213,         44.0375)
                (   1.35597,               0)
                (   1.35597,         18.5116)
                (   1.93213,         44.0375)
                (   1.35597,         18.5116)
                (  0.938142,               0)
                (  0.938142,         7.75767)
                (   1.35597,         18.5116)
                (  0.938142,         7.75767)
                (  0.636725,               0)
                (  0.636725,         3.20693)
                (  0.938142,         7.75767)
                (  0.636725,         3.20693)
                (  0.424314,               0)
                (  0.424314,         1.30944)
                (  0.636725,         3.20693)
                (  0.424314,         1.30944)
                (  0.277732,               0)
                (  0.277732,        0.527079)
                (  0.178979,        0.209401)
                (  0.113885,       0.0822499)
                ( 0.0717777,       0.0320216)
                ( 0.0449334,       0.0123862)
                ( 0.0279997,      0.00476976)
                ( 0.0173951,      0.00183133)
                (  0.010786,     0.000701786)
                (0.00667969,     0.000268604)
                (0.00413351,     0.000102728)
                (0.00255666,     3.92692e-05)
                (0.00158087,     1.50068e-05)
                };
                \addplot[color=blue, mark=square, only marks, mark size=3.pt] coordinates {(-2,0)(0,0)};
            \end{axis}
    \end{tikzpicture}
    \caption{Polinômio de 3$^o$ grau: $3x^3 + 2x^2$.}
    \label{fig:polinomial}
\end{figure}

\begin{figure}[hbt!]
    \begin{tikzpicture}
        \begin{axis} [axis lines=center]
            \addplot [domain=-3.05:3.05, smooth, thick] { exp(x^2 - 4) -1 };
            % \addplot [dashed, domain=-2.5:2.8, smooth, thick] { 2*x*exp(x^2-4) };
            \addplot[color=red, mark=o, mark size=2.pt] coordinates {
                (      3,         147.413)
                (    2.5,         8.48774)
                (2.46945,               0)
                (2.46945,         7.15144)
                (      3,         147.413)
                (2.46945,         7.15144)
                ( 2.4424,               0)
                ( 2.4424,         6.13723)
                (2.27871,               0)
                (2.27871,         2.29537)
                (2.18091,               0)
                (2.18091,         1.13053)
                (2.08599,        0.421002)
                (2.02967,        0.127008)
                (2.00534,       0.0216211)
                (2.00035,      0.00139638)
                (      2,      1.6707e-05)
                };
                \addplot[color=blue, mark=square, only marks, mark size=3.pt] coordinates {(-2,0)(2,0)};
            \end{axis}
        \end{tikzpicture}
        \caption{Função exponencial: $e^{x^2-4}-1$.}
        \label{fig:fexponencial}
    \end{figure}

    \begin{figure}[hbt!]
        \begin{tikzpicture}
        \begin{axis} [axis lines=center]
            \addplot[smooth, thick] table [x=x, y = fx] {f_trig.txt};
            % \addplot[smooth, thick, dashed] table [x=x, y = fpx] {f_trig.txt};
            \addplot[color=red, mark=o, mark size=2.pt] coordinates {
               (         27,         558.137)
               (         25,             700)
               (    34.8687,               0)
               (    34.8687,         1318.36)
               (         25,             700)
               (    34.8687,         1318.36)
               (    13.8284,               0)
               (    13.8284,         207.954)
               (    34.8687,         1318.36)
               (    13.8284,         207.954)
               (    9.88806,               0)
               (    9.88806,          29.785)
               (    13.8284,         207.954)
               (    9.88806,          29.785)
               (    9.22934,               0)
               (    9.22934,         32.5843)
               (    9.88806,          29.785)
               (    9.22934,         32.5843)
               (    16.8971,               0)
               (    16.8971,         246.243)
               (    9.22934,         32.5843)
               (    16.8971,         246.243)
               (    8.05996,               0)
               (    8.05996,         45.4676)
               (    16.8971,         246.243)
               (    8.05996,         45.4676)
               (    6.05871,               0)
               (    6.05871,         50.9709)
               (    8.05996,         45.4676)
               (    6.05871,         50.9709)
               (    24.5942,               0)
               (    24.5942,         668.878)
               (    6.05871,         50.9709)
               (    24.5942,         668.878)
               (    4.52973,               0)
               (    4.52973,         33.6631)
               (    24.5942,         668.878)
               (    4.52973,         33.6631)
               (    3.46642,               0)
               (    3.46642,         19.8354)
               (    4.52973,         33.6631)
               (    3.46642,         19.8354)
               (    1.94113,               0)
               (    1.94113,         7.05933)
               (    3.46642,         19.8354)
               (    1.94113,         7.05933)
               (    1.09834,               0)
               (    1.09834,         3.43304)
               (    1.94113,         7.05933)
               (    1.09834,         3.43304)
               (   0.300464,               0)
               (   0.300464,        0.902195)
               (    1.09834,         3.43304)
               (   0.300464,        0.902195)
               (  0.0160378,               0)
               (  0.0160378,       0.0481135)
               (   0.300464,        0.902195)
               (  0.0160378,       0.0481135)
               (1.50578e-05,     4.51734e-05)
               (1.50578e-05,     4.51734e-05)
               (2.04541e-12,     6.13622e-12)
                };
              \addplot[color=blue, mark=square, only marks, mark size=3.pt] coordinates {(0.,0)(-28.9569,0)(-21.2268,0)(-18.6878,0)(-11.6909,0)(-7.88405,0)(-3.63114,0)};
        \end{axis}
    \end{tikzpicture}
    \caption{Função trigonométrica: $3x + x^2 \sin^2 \frac{x\pi}{10}$. Pontos inciais: 27 e 25.}
    \label{fig:ftrigonometrica}
\end{figure}

\begin{figure}[hbt!]
    \begin{tikzpicture}
        \begin{axis} [axis lines=center]
            \addplot[smooth, thick] table [x=x, y = fx] {f_3Dwell.txt};
            % \addplot[smooth, thick, dashed] table [x=x, y = fpx] {f_3Dwell.txt};
            \addplot[color=blue, mark=square, only marks, mark size=3.pt] coordinates {(10.3144,0)};
            \addplot[color=red, mark=square, mark size=2.pt] coordinates {
                (14.7738,    -2.92074 )
                (10.8341,   -0.347228 )
                (10.3025,  0.00823089 )
                (10.3149, -0.000284164)
                (10.3144, -3.47599e-07)
                (10.3144,  1.45128e-11)
                };
        \end{axis}
    \end{tikzpicture}
    \caption{Função de erro de $\Delta S f(\alpha)$ para o caso Poço 3D.}
    \label{fig:fpoco3d}
\end{figure}

%% \section{}
%% \label{}

%% If you have bibdatabase file and want bibtex to generate the
%% bibitems, please use
%%

\bibliographystyle{elsarticle-num}
\bibliography{refs}

%% else use the following coding to input the bibitems directly in the
%% TeX file.

% \begin{thebibliography}{00}

%% \bibitem{label}
%% Text of bibliographic item

% \bibitem{}

% \end{thebibliography}

% \newpage
% \FloatBarrier
% \section{Código em C}

% O código de ambos métodos foi implementado em um único arquivo. O código é apresentado em duas partes neste documento para facilitar a leitura. O código pode ser encontrado em \href{https://github.com/TiagoCAAmorim/numerical-methods}{https://github.com/TiagoCAAmorim/numerical-methods}.

% \subsection{Método da Bissecção}
% \lstinputlisting[language=C, linerange={1-229}]{./02_newton_raphson.c}

% \subsection{Método de Newton-Raphson}
% \lstinputlisting[language=C, linerange={231-445}]{./02_newton_raphson.c}

% \subsection{Método da Mínima Curvatura}
% \lstinputlisting[language=C, linerange={448-958}]{./02_newton_raphson.c}

\end{document}
\endinput
\documentclass[final,5p]{elsarticle}

% \documentclass[preprint,12pt]{elsarticle}

%% Use the option review to obtain double line spacing
%% \documentclass[authoryear,preprint,review,12pt]{elsarticle}

%% Use the options 1p,twocolumn; 3p; 3p,twocolumn; 5p; or 5p,twocolumn
%% for a journal layout:
% \documentclass[final,1p,times]{elsarticle}
%% \documentclass[final,1p,times,twocolumn]{elsarticle}
% \documentclass[final,3p,times]{elsarticle}
%% \documentclass[final,3p,times,twocolumn]{elsarticle}
% \documentclass[final,5p,times]{elsarticle}
%% \documentclass[final,5p,times,twocolumn]{elsarticle}
\usepackage[portuguese]{babel}

%% For including figures, graphicx.sty has been loaded in
%% elsarticle.cls. If you prefer to use the old commands
%% please give \usepackage{epsfig}

%% The amssymb package provides various useful mathematical symbols
\usepackage{amssymb}
\usepackage{amsmath}
\usepackage{multirow}

\usepackage{pgfplots}
\pgfplotsset{compat=1.18}
\usepgfplotslibrary{statistics}
\usepackage{pgfplotstable}

\usepackage{placeins}
\usepackage{hyperref}
\numberwithin{equation}{section}

\usepackage{algorithm}
\usepackage[noEnd=true, indLines=true]{algpseudocodex}
\algrenewcommand\algorithmicrequire{\textbf{Entrada:}}
\algrenewcommand\algorithmicwhile{\textbf{Enquanto}}
\algrenewcommand\algorithmicrepeat{\textbf{Repete}}
\algrenewcommand\algorithmicuntil{\textbf{Até}}
\algrenewcommand\algorithmicif{\textbf{Se}}
\algrenewcommand\algorithmicthen{\textbf{então}}
\algrenewcommand\algorithmicelse{\textbf{Caso contrário}}
\algrenewcommand\algorithmicensure{\textbf{Objetivo:}}
\algrenewcommand\algorithmicreturn{\textbf{Retorna:}}
\algrenewcommand\algorithmicdo{\textbf{faça}}
\algrenewcommand\algorithmicforall{\textbf{Para todos}}
\algnewcommand{\LineComment}[1]{\State \(\triangleright\) \textcolor{black!50}{\emph{#1}}}

% \usepackage[fleqn]{nccmath}
% \usepackage{multicol}


%=========== Gloabal Tikz settings
% \pgfplotsset{compat=newest}
% \usetikzlibrary{math}
% \pgfplotsset{
%     height = 10cm,
%     width = 10cm,
%     tick pos = left,
%     legend style={at={(0.98,0.30)}, anchor=east},
%     legend cell align=left,
%     }
%  \pgfkeys{
%     /pgf/number format/.cd,
%     fixed,
%     precision = 1,
%     set thousands separator = {}
% }

%% The amsthm package provides extended theorem environments
%% \usepackage{amsthm}

%% The lineno packages adds line numbers. Start line numbering with
%% \begin{linenumbers}, end it with \end{linenumbers}. Or switch it on
%% for the whole article with \linenumbers.
%% \usepackage{lineno}

\usepackage{listings}
\usepackage{xcolor}

\definecolor{codegreen}{rgb}{0,0.6,0}
\definecolor{codegray}{rgb}{0.5,0.5,0.5}
\definecolor{codepurple}{rgb}{0.58,0,0.82}
\definecolor{backcolour}{rgb}{0.98,0.98,0.98}

\lstdefinestyle{mystyle}{
    backgroundcolor=\color{backcolour},
    commentstyle=\color{codegreen},
    keywordstyle=\color{magenta},
    numberstyle=\tiny\color{codegray},
    stringstyle=\color{codepurple},
    basicstyle=\ttfamily\footnotesize,
    breakatwhitespace=false,
    breaklines=true,
    captionpos=b,
    keepspaces=true,
    numbers=left,
    numbersep=5pt,
    showspaces=false,
    showstringspaces=false,
    showtabs=false,
    tabsize=2
}

\lstset{style=mystyle}

% \journal{Nuclear Physics B}

\begin{document}

\begin{frontmatter}

%% Title, authors and addresses

%% use the tnoteref command within \title for footnotes;
%% use the tnotetext command for theassociated footnote;
%% use the fnref command within \author or \address for footnotes;
%% use the fntext command for theassociated footnote;
%% use the corref command within \author for corresponding author footnotes;
%% use the cortext command for theassociated footnote;
%% use the ead command for the email address,
%% and the form \ead[url] for the home page:
%% \title{Title\tnoteref{label1}}
%% \tnotetext[label1]{}
%% \author{Name\corref{cor1}\fnref{label2}}
%% \ead{email address}
%% \ead[url]{home page}
%% \fntext[label2]{}
%% \cortext[cor1]{}
%% \affiliation{organization={},
%%             addressline={},
%%             city={},
%%             postcode={},
%%             state={},
%%             country={}}
%% \fntext[label3]{}

\title{Aplicação de Métodos de Múltiplo Passo para Resolver o Comportamento de Aquíferos Analíticos\tnoteref{label_title}}
\tnotetext[label_title]{Relatório número 8 como parte dos requisitos da disciplina IM253: Métodos Numéricos para Fenômenos de Transporte.}

%% use optional labels to link authors explicitly to addresses:
%% \author[label1,label2]{}
%% \affiliation[label1]{organization={},
%%             addressline={},
%%             city={},
%%             postcode={},
%%             state={},
%%             country={}}
%%
%% \affiliation[label2]{organization={},
%%             addressline={},
%%             city={},
%%             postcode={},
%%             state={},
%%             country={}}

\author{Tiago C. A. Amorim\fnref{label_author}}
\tnotetext[label_author]{Atualmente cursando doutorado no Departamento de Engenharia de Petróleo da Faculdade de Engenharia Mecânica da UNICAMP (Campinas/SP, Brasil).}
\ead{t100675@dac.unicamp.br}
\affiliation[Tiago C. A. Amorim]{organization={Petrobras},%Department and Organization
addressline={Av. Henrique Valadares, 28},
city={Rio de Janeiro},
postcode={20231-030},
state={RJ},
country={Brasil}}

\begin{abstract}

    O aquífero analítico proposto por Fetkovich é de grande utilidade na caracterização do comportamento de reservatórios de petróleo. Foram avaliados métodos de aproximação numérica de problemas de valor inicial para resolver este modelo de aquífero. Os métodos de múltiplos passos tiveram bom desempenho na previsão do comportamento do aquífero analítico testado. Os métodos de Adams-Bashforth e Adams-Moulton foram melhores que os métodos de Runge-Kutta de ordem equivalente e que o método de Fetkovich. Entre os métodos implícitos e explícitos avaliados, o método explícito de maior ordem (Adams-Bashforth de 5 passos) foi que teve os melhores resultados.

\end{abstract}


%%Graphical abstract
% \begin{graphicalabstract}
%\includegraphics{grabs}
% \end{graphicalabstract}

%%Research highlights
% \begin{highlights}
% \item Research highlight 1
% \item Research highlight 2
% \end{highlights}

\begin{keyword}
    Método de Fetkovich \sep Métodos de Runge-Kutta \sep Métodos de Adams \sep Fluxo em Meio Poroso
%% keywords here, in the form: keyword \sep keyword

%% PACS codes here, in the form: \PACS code \sep code

%% MSC codes here, in the form: \MSC code \sep code
%% or \MSC[2008] code \sep code (2000 is the default)

\end{keyword}

\end{frontmatter}

%% \linenumbers

%% main text
\section{Introdução}

    Este trabalho se propõe a resolver o modelo de aquífero numérico proposto por Fetkovich com métodos de aproximação numérica de um problema de valor inicial. Avaliações anteriores mostraram que o método de Euler não é adequado para resolver o problema proposto. Entre os métodos de Runge-Kutta testados ($2^a$, $3^a$ e $4^a$ ordem), apenas o de quarta ordem teve resultados próximos daqueles do método de Fetkovich \cite{relatoriorungekutta}.

    Neste relatório é avaliada a aplicação dos métodos de múltiplos passos: Adams-Bashforth (implícitos) e Adams-Moulton (explícitos), de diferentes ordens.

\section{Metodologia}

        Nos relatórios anteriores \cite{relatorioeuler}\cite{relatoriorungekutta} foi feita uma descrição dos métodos de Fetkovich, Euler, Aitken e Runge-Kutta. Também foi apresentada a formalização do modelo de aquífero proposto por Fetkovich como um problema de valor inicial (PVI). Este relatório focará nas implementações adicionais que foram feitas e seus resultados.

    \subsection{Métodos de Passo Múltiplo}

        Os métodos de passo múltiplo fazem uso de mais de uma das aproximações de $w_j \approx y(t_j)$ de para estimar o próximo o próximo ponto da malha. Dado um problema de valor inicial do tipo:

        \begin{align}
            \frac{dy}{dt} = f(t,y), \quad a \leq t \leq b, \quad y(a) = \alpha \label{eq:pvi}
        \end{align}

        Um método de passo múltiplo buscará uma aproximação para $w_{j+1}$ em $t_{j+1}$ com a seguinte forma:

        \begin{align}
            w_{j+1} =& \sum_{i=0}^{m-1} \left( a_{m-1-i} w_{j-i} \right) \nonumber \\
            &+ h \sum_{i=0}^{m} \left[ b_{m-i} f(t_{j+1-i},w_{j+1-i}) \right] \label{eq:passomultiplo}
        \end{align}
        com
        \begin{align}
            j&=m-1,m,\ldots,n-1 \nonumber \\
            h&=(b-a)/n \nonumber
        \end{align}
        e os valores iniciais: $w_0=\alpha, w_1=\alpha_1, \; \ldots \; w_{m-1}=\alpha_{m-1}$.

        Observa-se que se $b_m \neq 0$ o próprio valor de $f(t_{j+1},w_{j+1})$ pode ser usado para estimar $w_{j+1}$. Os métodos com esta característica são chamados de implícitos. Se $b_m = 0$ os métodos são chamados de explícitos.

        Foram implementados diferentes métodos explícitos de Adams-Bashforth e implícitos de Adams-Moulton \cite{burden2016analise}.

    \subsubsection{Métodos Explícitos}

        Método explícito de Adams-Bashforth de dois passos:

        \begin{align}
            w_0 &= \alpha \nonumber \\
            w_1 &= \alpha_1 \nonumber \\
            w_{j+1} &= w_j + \frac{h}{2} \left[ 3 f(t_j, w_j) - f(t_{j-1}, w_{j-1}) \right] \label{eq:adamsbashforth2}
        \end{align}

        \noindent com $j=1,2,\ldots,n-1$

        \noindent erro de truncamento local: $\tau_{j+1}(h) = \frac{5}{12}y^{(3)}(\mu_j)h^2$, com \\ $\mu_j \, \epsilon \, (t_j,t_{j+1})$

        Para estimar $\alpha_1$ foi utilizado um método de Runge-Kutta de $2^a$ ordem, que tem erro de truncamento local de mesma ordem: $\mathcal{O}(h^2)$.

        Método explícito de Adams-Bashforth de três passos:

        \begin{align}
            w_0 =& \alpha \nonumber \\
            w_1 =& \alpha_1 \nonumber \\
            w_2 =& \alpha_2 \nonumber \\
            w_{j+1} =& w_j + \frac{h}{12} \left[ 23 f(t_j, w_j) - 16 f(t_{j-1}, w_{j-1}) \right. \nonumber \\
            & \left. + 5 f(t_{j-2}, w_{j-2}) \right] \label{eq:adamsbashforth3}
        \end{align}

        \noindent com $j=2,3,\ldots,n-1$

        \noindent erro de truncamento local: $\tau_{j+1}(h) = \frac{3}{8}y^{(4)}(\mu_j)h^3$, com \\ $\mu_j \, \epsilon \, (t_j,t_{j+1})$

        Para estimar $\alpha_1$ e $\alpha_2$ foi utilizado um método de Runge-Kutta de $3^a$ ordem, que tem erro de truncamento local de mesma ordem: $\mathcal{O}(h^3)$.

        Método explícito de Adams-Bashforth de quatro passos:

        \begin{align}
            w_0 &= \alpha \nonumber \\
            w_1 &= \alpha_1 \nonumber \\
            w_2 &= \alpha_2 \nonumber \\
            w_3 &= \alpha_3 \nonumber \\
            w_{j+1} &= w_j + \frac{h}{24} \left[ 55 f(t_j, w_j) - 59 f(t_{j-1}, w_{j-1}) \right. \nonumber \\
            & \left. + 37 f(t_{j-2}, w_{j-2}) - 9 f(t_{j-3}, w_{j-3}) \right] \label{eq:adamsbashforth4}
        \end{align}

        \noindent com $j=3,4,\ldots,n-1$

        \noindent erro de truncamento local: $\tau_{j+1}(h) = \frac{251}{720}y^{(5)}(\mu_j)h^4$, com \\ $\mu_j \, \epsilon \, (t_j,t_{j+1})$

        Para estimar $\alpha_1$, $\alpha_2$ e $\alpha_3$ foi utilizado um método de Runge-Kutta de $4^a$ ordem, que tem erro de truncamento local de mesma ordem: $\mathcal{O}(h^4)$.

        Método explícito de Adams-Bashforth de cinco passos:

        \begin{align}
            w_0 &= \alpha \nonumber \\
            w_1 &= \alpha_1 \nonumber \\
            w_2 &= \alpha_2 \nonumber \\
            w_3 &= \alpha_3 \nonumber \\
            w_4 &= \alpha_4 \nonumber \\
            w_{j+1} &= w_j + \frac{h}{720} \left[ 1901 f(t_j, w_j) - 2774 f(t_{j-1}, w_{j-1}) \right. \nonumber \\
            & + 2616 f(t_{j-2}, w_{j-2}) - 1274 f(t_{j-3}, w_{j-3}) \nonumber \\
            &\left.+ 251 f(t_{j-4}, w_{j-4})\right] \label{eq:adamsbashforth5}
        \end{align}

        \noindent com $j=4,5,\ldots,n-1$

        \noindent erro de truncamento local: $\tau_{j+1}(h) = \frac{95}{288}y^{(6)}(\mu_j)h^5$, com \\ $\mu_j \, \epsilon \, (t_j,t_{j+1})$

        Para estimar $\alpha_1$, $\alpha_2$, $\alpha_3$ e $\alpha_4$ foi utilizado um método de Runge-Kutta de ordem superior, que tem erro de truncamento local de mesma ordem: $\mathcal{O}(h^5)$.

    \subsubsection{Métodos Implícitos}

        Método implícito de Adams-Moulton de um passo:

        \begin{align}
            w_0 &= \alpha \nonumber \\
            w_{j+1} &= w_j + \frac{h}{2} \left[ f(t_{j+1}, w_{j+1}) + f(t_j, w_j) \right] \label{eq:adamsmoulton1}
        \end{align}

        \noindent com $j=0,1,\ldots,n-1$

        \noindent erro de truncamento local: $\tau_{j+1}(h) = -\frac{1}{12}y^{(3)}(\mu_j)h^2$, com \\ $\mu_j \, \epsilon \, (t_j,t_{j+1})$

        Método implícito de Adams-Moulton de dois passos:

        \begin{align}
            w_0 &= \alpha \nonumber \\
            w_1 &= \alpha_1 \nonumber \\
            w_{j+1} &= w_j + \frac{h}{12} \left[ 5 f(t_{j+1}, w_{j+1}) + 8 f(t_j, w_j) \right. \nonumber \\
            &\left. - f(t_{j-1}, w_{j-1}) \right] \label{eq:adamsmoulton2}
        \end{align}

        \noindent com $j=1,2,\ldots,n-1$

        \noindent erro de truncamento local: $\tau_{j+1}(h) = -\frac{1}{24}y^{(4)}(\mu_j)h^3$, com \\ $\mu_j \, \epsilon \, (t_j,t_{j+1})$

        Método implícito de Adams-Moulton de três passos:

        \begin{align}
            w_0 &= \alpha \nonumber \\
            w_1 &= \alpha_1 \nonumber \\
            w_2 &= \alpha_2 \nonumber \\
            w_{j+1} &= w_j + \frac{h}{24} \left[ 9 f(t_{j+1}, w_{j+1}) + 19 f(t_j, w_j) \right. \nonumber \\
            &\left. - 5 f(t_{j-1}, w_{j-1}) +f(t_{j-2}, w_{j-2}) \right] \label{eq:adamsmoulton3}
        \end{align}

        \noindent com $j=2,3,\ldots,n-1$

        \noindent erro de truncamento local: $\tau_{j+1}(h) = -\frac{19}{720}y^{(5)}(\mu_j)h^4$, com \\ $\mu_j \, \epsilon \, (t_j,t_{j+1})$

        Método implícito de Adams-Moulton de quatro passos:

        \begin{align}
            w_0 &= \alpha \nonumber \\
            w_1 &= \alpha_1 \nonumber \\
            w_2 &= \alpha_2 \nonumber \\
            w_3 &= \alpha_3 \nonumber \\
            w_{j+1} &= w_j + \frac{h}{720} \left[ 251 f(t_{j+1}, w_{j+1}) + 646 f(t_j, w_j) \right. \nonumber \\
            & - 264 f(t_{j-1}, w_{j-1}) + 106 f(t_{j-2}, w_{j-2}) \nonumber \\
            &\left. - 19 f(t_{j-3}, w_{j-3}) \right] \label{eq:adamsmoulton4}
        \end{align}

        \noindent com $j=3,4,\ldots,n-1$

        \noindent erro de truncamento local: $\tau_{j+1}(h) = -\frac{3}{160}y^{(6)}(\mu_j)h^5$, com \\ $\mu_j \, \epsilon \, (t_j,t_{j+1})$

        Para todos os métodos implícitos foi implementado o esquema \emph{preditor-corretor}. O método explícito equivalente (de mesmo $\mathcal{O}(h^n)$) é utilizado para fazer a primeira estimativa de $w_{j+1}$ (preditor). Em seguida o método implícito é utilizado para atualizar o valor de $w_{j+1}$ antes de avançar para o próximo passo de temp (corretor).

        Para estimar os $\alpha_i$ foi utilizada a mesma lógica dos métodos explícitos de usar métodos de Runge-Kutta com erro de truncamento local de mesma ordem.

    \subsection{Método de Runge-Kutta de Ordem Superior}

        Para a implementação do método de Adams-Bashforth de cinco passos foi necessário utilizar um dos métodos de Runge-Kutta com erro de aproximação equivalente, e que não foi discutido no relatório anterior. Foi implementada uma das fórmulas do método de Runge-Kutta-Fehlberg \cite{burden2016analise}\footnote{Este é um método adaptativo, que usa formas parecidas do método de Runge-Kutta para estimar erros de aproximação e definir o tamanho do passo de tempo. Foi implementada apenas uma das fórmulas deste método, e utilizada com tamanho de passo de tempo constante.}:

        \begin{align}
            w_0 =& \alpha \nonumber \\
            k_1 =& hf\left(t_j,w_j\right) \nonumber \\
            k_2 =& hf\left(t_j + \frac{h}{4},w_j + \frac{1}{4}k_1 \right) \nonumber \\
            k_3 =& hf\left(t_j + \frac{3h}{8},w_j + \frac{3}{32}k_1+\frac{9}{32}k_2 \right) \nonumber \\
            k_4 =& hf\left(t_j + \frac{12h}{13},w_j + \frac{1932}{2197}k_1 - \frac{7200}{2197}k_2+\frac{7296}{2197}k_3\right) \nonumber \\
            k_5 =& hf\left(t_j + h,w_j + \frac{439}{216}k_1 - 8k_2+\frac{3680}{513}k_3 \right. \nonumber \\
            & \left. - \frac{845}{4104}k_4 \right) \nonumber \\
            k_6 =& hf\left(t_j + \frac{h}{2},w_j - \frac{8}{27}k_1 + 2k_2 - \frac{3544}{2565}k_3+\frac{1859}{4104}k_4 \right. \nonumber \\
            & \left. - \frac{11}{40}k_5 \right) \nonumber \\
            w_{j+1} =& w_{j} + \frac{16}{135} k_1 + \frac{6656}{12825} k_3 + \frac{28561}{56430} k_4 - \frac{9}{20} k_5 \nonumber \\
            & + \frac{2}{55} k_6 \label{eq:rk6} \\
            \text{com } j&=0,1,\ldots,n-1 \nonumber
        \end{align}

        Esta versão do método de Runge-Kutta tem erro de truncamento local $\mathcal{O}(h^5)$.

\section{Implementação} \label{sec:implementacao}

        Todo o código utilizado nesta análise foi desenvolvido em C++\footnote{Alguns detalhes de implementação foram suprimidos neste relatório para facilitar a leitura. Para maiores informações verificar o relatório anterior \cite{relatoriorungekutta}.}. Foram criados objetos próprios para cada elemento integrante do problema proposto:

        \begin{description}
            \item[IVP] Classe que define um problema de valor inicial na forma \ref{eq:pvi}.
            \begin{itemize}
                \item O usuário precisa especificar $f(t,y)$, $a$ (tempo inicial), $b$ (tempo final), $n$ (número de passos de tempo) e $y(a)$ (valor inicial).
                \item O usuário também pode especificar a solução exata ($y=f(t)$), para calcular o erro de aproximação.
                \item São disponibilizados diferentes métodos de resolução do PVI: Euler, Runge-Kutta, Adams-Bashforth e Adams-Moulton.
                \item Foi implementada também uma rotina que tenta melhorar as respostas de Euler e Runge-Kutta utilizando o método de Aitken.
            \end{itemize}
            \item[Fetkovich] Classe que resolve o comportamento de um aquífero como proposto por Fetkovich.
            \begin{itemize}
                \item O usuário precisa definir as características do aquífero e prover uma função que retorne a pressão na interface do aquífero com o reservatório. Esta função depende do tempo e do influxo acumulado de água do aquífero para o reservatório ($W_e$).
            \end{itemize}
        \end{description}

\section{Resultados}

        Foram utilizados os mesmos problemas de valor inicial do relatório anterior para comparar os resultados do método de Runge-Kutta com os de Adams-Bashforth e Adams-Moulton:

        \begin{enumerate}
            \item $y' = y - t^2 + 1, \quad 0 \leq t \leq  2, \quad y(0) = 0.5$ \label{item:pvi1}
            \item $y' = -2y + 3 e^t, \quad 0 \leq t \leq  2, \quad y(0) = 3.0$ \label{item:pvi2}
            \item $y' = 4 cos(t) - 8 sin(t) + 2 y, \quad 0 \leq t \leq  2, \quad y(0) = 3.0$ \label{item:pvi3}
        \end{enumerate}

        Os resultados exatos de cada problema são, respectivamente:

        \begin{enumerate}
            \item $y = (t+1)^2 - 0.5 e^t$
            \item $y = 2 e^{-2 t} + e^t$
            \item $y = 4 sin(t) + 3 e^{2 t}$
        \end{enumerate}

        Foi aplicada a mesma lógica utilizada no relatório anterior, em que foi assumido que em uma aplicação mais computacionalmente intensiva a avaliação da função $f(t,y)$ tem o maior peso. Todas as avaliações foram feitas de forma a que o número de chamadas à função $f(t,y)$ fosse aproximadamente a mesma. Desta forma, as estimativas com os métodos de Runge-Kutta tem menos passos de tempo que as estimativas com os métodos de Adams-Moulton, que por sua vez tem menos passos de tempo que as estimativas com os métodos de Adams-Bashforth. Com cada método foram realizadas aproximadamente 200 avaliações da função $f(t,y)$.

        Observa-se nos gráficos de erro relativo (figuras \ref{fig:teste1}, \ref{fig:teste2} e \ref{fig:teste3}) que os métodos de passo múltiplo conseguiram melhores resultados que os métodos de Runge-Kutta equivalentes (Runge-Kutta de $4^a$ ordem comparado com Adams-Molton de 3 passos e Adams-Bashforth de 4 passos, e Runge-Kutta de $5^a$ ordem comparado com Adams-Molton de 4 passos e Adams-Bashforth de 5 passos). Os resultados dos métodos de passo múltiplo explícitos e implicitos de mesma ordem ficaram próximos, invertendo a posição entre os diferentes testes.

        \begin{figure}[hbt!]
            \begin{tikzpicture}
                \begin{semilogyaxis}[
                    grid=both,
                    xlabel = {$t$},
                    ylabel = {$|w_i-y(t)|/|y(t)|$},
                    legend style={at={(1.03,0.05)}, anchor=east, font=\footnotesize},
                    ]
                    \addplot[color=black, solid, smooth, thick] table [x=t, y=exact_error] {test1_rungekutta4.txt};
                    \addplot[color=blue, solid, smooth, thick] table [x=t, y=exact_error] {test1_rungekutta5.txt};
                    \addplot[color=red, solid, smooth, thick] table [x=t, y=exact_error] {test1_adams_1_imp.txt};
                    \addplot[color=red, dashed, smooth, thick] table [x=t, y=exact_error] {test1_adams_2_exp.txt};
                    \addplot[color=green, solid, smooth, thick] table [x=t, y=exact_error] {test1_adams_2_imp.txt};
                    \addplot[color=green, dashed, smooth, thick] table [x=t, y=exact_error] {test1_adams_3_exp.txt};
                    \addplot[color=cyan, solid, smooth, thick] table [x=t, y=exact_error] {test1_adams_3_imp.txt};
                    \addplot[color=cyan, dashed, smooth, thick] table [x=t, y=exact_error] {test1_adams_4_exp.txt};
                    \addplot[color=yellow, solid, smooth, thick] table [x=t, y=exact_error] {test1_adams_4_imp.txt};
                    \addplot[color=yellow, dashed, smooth, thick] table [x=t, y=exact_error] {test1_adams_5_exp.txt};
                    \legend{Runge-K. $4^a$, Runge-K. $5^a$, Adams-M. 1p, Adams-B. 2p, Adams-M. 2p, Adams-B. 3p, Adams-M. 3p, Adams-B. 4p, Adams-M. 4p, Adams-B. 5p};
                \end{semilogyaxis}
            \end{tikzpicture}
            \caption{Erro da aproximação com os métodos de Runge-Kutta, Adams-Bashforth e Adams-Moulton, do PVI número \ref{item:pvi1}.}
            \label{fig:teste1}
        \end{figure}

        \begin{figure}[hbt!]
            \begin{tikzpicture}
                \begin{semilogyaxis}[
                    grid=both,
                    xlabel = {$t$},
                    ylabel = {$|w_i-y(t)|/|y(t)|$},
                    legend style={at={(1.03,-0.05)}, anchor=east, font=\footnotesize},
                    ]
                    \addplot[color=black, solid, smooth, thick] table [x=t, y=exact_error] {test2_rungekutta4.txt};
                    \addplot[color=blue, solid, smooth, thick] table [x=t, y=exact_error] {test2_rungekutta5.txt};
                    \addplot[color=red, solid, smooth, thick] table [x=t, y=exact_error] {test2_adams_1_imp.txt};
                    \addplot[color=red, dashed, smooth, thick] table [x=t, y=exact_error] {test2_adams_2_exp.txt};
                    \addplot[color=green, solid, smooth, thick] table [x=t, y=exact_error] {test2_adams_2_imp.txt};
                    \addplot[color=green, dashed, smooth, thick] table [x=t, y=exact_error] {test2_adams_3_exp.txt};
                    \addplot[color=cyan, solid, smooth, thick] table [x=t, y=exact_error] {test2_adams_3_imp.txt};
                    \addplot[color=cyan, dashed, smooth, thick] table [x=t, y=exact_error] {test2_adams_4_exp.txt};
                    \addplot[color=yellow, solid, smooth, thick] table [x=t, y=exact_error] {test2_adams_4_imp.txt};
                    \addplot[color=yellow, dashed, smooth, thick] table [x=t, y=exact_error] {test2_adams_5_exp.txt};
                    \legend{Runge-K. $4^a$, Runge-K. $5^a$, Adams-M. 1p, Adams-B. 2p, Adams-M. 2p, Adams-B. 3p, Adams-M. 3p, Adams-B. 4p, Adams-M. 4p, Adams-B. 5p};
                \end{semilogyaxis}
            \end{tikzpicture}
            \caption{Erro da aproximação com os métodos de Runge-Kutta, Adams-Bashforth e Adams-Moulton, do PVI número \ref{item:pvi2}.}
            \label{fig:teste2}
        \end{figure}

        \begin{figure}[hbt!]
            \begin{tikzpicture}
                \begin{semilogyaxis}[
                    grid=both,
                    xlabel = {$t$},
                    ylabel = {$|w_i-y(t)|/|y(t)|$},
                    legend style={at={(1.03,0.10)}, anchor=east, font=\footnotesize},
                    ]
                    \addplot[color=black, solid, smooth, thick] table [x=t, y=exact_error] {test3_rungekutta4.txt};
                    \addplot[color=blue, solid, smooth, thick] table [x=t, y=exact_error] {test3_rungekutta5.txt};
                    \addplot[color=red, solid, smooth, thick] table [x=t, y=exact_error] {test3_adams_1_imp.txt};
                    \addplot[color=red, dashed, smooth, thick] table [x=t, y=exact_error] {test3_adams_2_exp.txt};
                    \addplot[color=green, solid, smooth, thick] table [x=t, y=exact_error] {test3_adams_2_imp.txt};
                    \addplot[color=green, dashed, smooth, thick] table [x=t, y=exact_error] {test3_adams_3_exp.txt};
                    \addplot[color=cyan, solid, smooth, thick] table [x=t, y=exact_error] {test3_adams_3_imp.txt};
                    \addplot[color=cyan, dashed, smooth, thick] table [x=t, y=exact_error] {test3_adams_4_exp.txt};
                    \addplot[color=yellow, solid, smooth, thick] table [x=t, y=exact_error] {test3_adams_4_imp.txt};
                    \addplot[color=yellow, dashed, smooth, thick] table [x=t, y=exact_error] {test3_adams_5_exp.txt};
                    \legend{Runge-K. $4^a$, Runge-K. $5^a$, Adams-M. 1p, Adams-B. 2p, Adams-M. 2p, Adams-B. 3p, Adams-M. 3p, Adams-B. 4p, Adams-M. 4p, Adams-B. 5p};
                \end{semilogyaxis}
            \end{tikzpicture}
            \caption{Erro da aproximação com os métodos de Runge-Kutta, Adams-Bashforth e Adams-Moulton, do PVI número \ref{item:pvi3}.}
            \label{fig:teste3}
        \end{figure}

        O mesmo teste foi feito na resolução do modelo de aquífero de Fetkovich. O número exato de avaliações com o método de Fetkovich foi 233, enquanto os demais métodos usaram aproximadamente 200 avaliações. A figura \ref{fig:testeAq} mostra que apenas os métodos de Adams-Moulton de 1 passo e Adams-Bashforth de 2 passos não conseguiram desempenho melhor que o método de Fetkovich. Foi feita a comparação em termos de produção acumulada de água (figura \ref{fig:testeAqWe}), e o método de Fetkovich se mostra com menores erros e mais estável que todos os demais. Contudo, os erros com os métodos de $\mathcal{O}(h^3)$ são quase todos menores que os de Fetkovich.

        \begin{figure}[hbt!]
            \begin{tikzpicture}
                \begin{semilogyaxis}[
                    grid=both,
                    xlabel = {$t$},
                    ylabel = {$|w_i-y(t)|/|y(t)|$},
                    legend style={at={(1.03,0.05)}, anchor=east, font=\footnotesize},
                    ]
                    \addplot[color=orange, solid, smooth, thick] table [x=Time, y=ErrorRel] {aq1_fetkovich.txt};
                    \addplot[color=black, solid, smooth, thick] table [x=t, y=exact_error] {aq1_rungekutta4.txt};
                    \addplot[color=blue, solid, smooth, thick] table [x=t, y=exact_error] {aq1_rungekutta5.txt};
                    \addplot[color=red, solid, smooth, thick] table [x=t, y=exact_error] {aq1_adams_1_imp.txt};
                    \addplot[color=red, dashed, smooth, thick] table [x=t, y=exact_error] {aq1_adams_2_exp.txt};
                    \addplot[color=green, solid, smooth, thick] table [x=t, y=exact_error] {aq1_adams_2_imp.txt};
                    \addplot[color=green, dashed, smooth, thick] table [x=t, y=exact_error] {aq1_adams_3_exp.txt};
                    \addplot[color=cyan, solid, smooth, thick] table [x=t, y=exact_error] {aq1_adams_3_imp.txt};
                    \addplot[color=cyan, dashed, smooth, thick] table [x=t, y=exact_error] {aq1_adams_4_exp.txt};
                    \addplot[color=yellow, solid, smooth, thick] table [x=t, y=exact_error] {aq1_adams_4_imp.txt};
                    \addplot[color=yellow, dashed, smooth, thick] table [x=t, y=exact_error] {aq1_adams_5_exp.txt};
                    \legend{Fetkovich, Runge-K. $4^a$, Runge-K. $5^a$, Adams-M. 1p, Adams-B. 2p, Adams-M. 2p, Adams-B. 3p, Adams-M. 3p, Adams-B. 4p, Adams-M. 4p, Adams-B. 5p};
                \end{semilogyaxis}
            \end{tikzpicture}
            \caption{Erro da aproximação da vazão de água com os métodos de Fetkovich, Runge-Kutta, Adams-Bashforth e Adams-Moulton.}
            \label{fig:testeAq}
        \end{figure}

        \begin{figure}[hbt!]
            \begin{tikzpicture}
                \begin{semilogyaxis}[
                    grid=both,
                    xlabel = {$t$},
                    ylabel = {$|\int w_i- \int y(t)|/|\int y(t)|$},
                    legend style={at={(1.03,-0.25)}, anchor=east, font=\footnotesize},
                    ]
                    \addplot[color=orange, solid, smooth, thick] table [x=Time, y=ErrorRelWe] {aq1_fetkovich.txt};
                    \addplot[color=black, solid, smooth, thick] table [x=t, y=S_error] {aq1_rungekutta4.txt};
                    \addplot[color=blue, solid, smooth, thick] table [x=t, y=S_error] {aq1_rungekutta5.txt};
                    \addplot[color=red, solid, smooth, thick] table [x=t, y=S_error] {aq1_adams_1_imp.txt};
                    \addplot[color=red, dashed, smooth, thick] table [x=t, y=S_error] {aq1_adams_2_exp.txt};
                    \addplot[color=green, solid, smooth, thick] table [x=t, y=S_error] {aq1_adams_2_imp.txt};
                    \addplot[color=green, dashed, smooth, thick] table [x=t, y=S_error] {aq1_adams_3_exp.txt};
                    \addplot[color=cyan, solid, smooth, thick] table [x=t, y=S_error] {aq1_adams_3_imp.txt};
                    \addplot[color=cyan, dashed, smooth, thick] table [x=t, y=S_error] {aq1_adams_4_exp.txt};
                    \addplot[color=yellow, solid, smooth, thick] table [x=t, y=S_error] {aq1_adams_4_imp.txt};
                    \addplot[color=yellow, dashed, smooth, thick] table [x=t, y=S_error] {aq1_adams_5_exp.txt};
                    \legend{Fetkovich, Runge-K. $4^a$, Runge-K. $5^a$, Adams-M. 1p, Adams-B. 2p, Adams-M. 2p, Adams-B. 3p, Adams-M. 3p, Adams-B. 4p, Adams-M. 4p, Adams-B. 5p};
                \end{semilogyaxis}
            \end{tikzpicture}
            \caption{Erro da aproximação da vazão de água acumulada com os métodos de Fetkovich, Runge-Kutta, Adams-Bashforth e Adams-Moulton.}
            \label{fig:testeAqWe}
        \end{figure}

        Assim como no relatório passado, foi feita uma sensibilidade do erro de estimativa da vazão de água acumulada dos diferentes métodos testados em função do número de avaliações de $f(t,y)$. Os gráficos foram montados em função do erro relativo no tempo final (Figura \ref{fig:testeAqSensEnd}) e do máximo erro relativo entre todos os time-steps (Figura \ref{fig:testeAqSensMax}). Os resultados mostram que o método de Fetkovich tem melhores resultados apenaas para um número reduzido de time-steps. Entre os métodos avaliados o de Adams-Bashforth de 5 passos foi o que mostrou melhor desempenho, tanto em termos de erro no tempo final como de erro relativo máximo.

        \begin{figure}[hbt!]
            \begin{tikzpicture}
                \begin{semilogyaxis}[
                    grid=both,
                    xlabel = {Avaliações de $f(t,y)$},
                    ylabel = {$|\int w_n - \int y(t_{final})|/|\int y(t_{final})|$},
                    xmax=1000,
                    legend style={at={(1.03,1.00)}, anchor=east, font=\footnotesize},
                    ]
                    \addplot[color=orange, solid, smooth, thick] table [x=Evaluations, y=ErrorEnd] {aq1_fetkovich_sens.txt};
                    \addplot[color=black, solid, smooth, thick] table [x=Evaluations, y=ErrorEnd] {aq1_rungekutta4_sens.txt};
                    \addplot[color=blue, solid, smooth, thick] table [x=Evaluations, y=ErrorEnd] {aq1_rungekutta5_sens.txt};
                    \addplot[color=red, solid, smooth, thick] table [x=Evaluations, y=ErrorEnd] {aq1_adams_1_imp_sens.txt};
                    \addplot[color=red, dashed, smooth, thick] table [x=Evaluations, y=ErrorEnd] {aq1_adams_2_exp_sens.txt};
                    \addplot[color=green, solid, smooth, thick] table [x=Evaluations, y=ErrorEnd] {aq1_adams_2_imp_sens.txt};
                    \addplot[color=green, dashed, smooth, thick] table [x=Evaluations, y=ErrorEnd] {aq1_adams_3_exp_sens.txt};
                    \addplot[color=cyan, solid, smooth, thick] table [x=Evaluations, y=ErrorEnd] {aq1_adams_3_imp_sens.txt};
                    \addplot[color=cyan, dashed, smooth, thick] table [x=Evaluations, y=ErrorEnd] {aq1_adams_4_exp_sens.txt};
                    \addplot[color=yellow, solid, smooth, thick] table [x=Evaluations, y=ErrorEnd] {aq1_adams_4_imp_sens.txt};
                    \addplot[color=yellow, dashed, smooth, thick] table [x=Evaluations, y=ErrorEnd] {aq1_adams_5_exp_sens.txt};
                    \legend{Fetkovich, Runge-K. $4^a$, Runge-K. $5^a$, Adams-M. 1p, Adams-B. 2p, Adams-M. 2p, Adams-B. 3p, Adams-M. 3p, Adams-B. 4p, Adams-M. 4p, Adams-B. 5p};
                \end{semilogyaxis}
            \end{tikzpicture}
            \caption{Sensibilidade do erro relativo da aproximação da vazão de água acumulada no tempo final com os métodos de Fetkovich, Runge-Kutta, Adams-Bashforth e Adams-Moulton.}
            \label{fig:testeAqSensEnd}
        \end{figure}

        \begin{figure}[hbt!]
            \begin{tikzpicture}
                \begin{semilogyaxis}[
                    grid=both,
                    xlabel = {Avaliações de $f(t,y)$},
                    ylabel = {$|\int w_j - \int y(t_j)|/|\int y(t_j)|_{max}$},
                    xmax=1000,
                    legend style={at={(1.03,0.80)}, anchor=east, font=\footnotesize},
                    ]
                    \addplot[color=orange, solid, smooth, thick] table [x=Evaluations, y=ErrorMax] {aq1_fetkovich_sens.txt};
                    \addplot[color=black, solid, smooth, thick] table [x=Evaluations, y=ErrorMax] {aq1_rungekutta4_sens.txt};
                    \addplot[color=blue, solid, smooth, thick] table [x=Evaluations, y=ErrorMax] {aq1_rungekutta5_sens.txt};
                    \addplot[color=red, solid, smooth, thick] table [x=Evaluations, y=ErrorMax] {aq1_adams_1_imp_sens.txt};
                    \addplot[color=red, dashed, smooth, thick] table [x=Evaluations, y=ErrorMax] {aq1_adams_2_exp_sens.txt};
                    \addplot[color=green, solid, smooth, thick] table [x=Evaluations, y=ErrorMax] {aq1_adams_2_imp_sens.txt};
                    \addplot[color=green, dashed, smooth, thick] table [x=Evaluations, y=ErrorMax] {aq1_adams_3_exp_sens.txt};
                    \addplot[color=cyan, solid, smooth, thick] table [x=Evaluations, y=ErrorMax] {aq1_adams_3_imp_sens.txt};
                    \addplot[color=cyan, dashed, smooth, thick] table [x=Evaluations, y=ErrorMax] {aq1_adams_4_exp_sens.txt};
                    \addplot[color=yellow, solid, smooth, thick] table [x=Evaluations, y=ErrorMax] {aq1_adams_4_imp_sens.txt};
                    \addplot[color=yellow, dashed, smooth, thick] table [x=Evaluations, y=ErrorMax] {aq1_adams_5_exp_sens.txt};
                    \legend{Fetkovich, Runge-K. $4^a$, Runge-K. $5^a$, Adams-M. 1p, Adams-B. 2p, Adams-M. 2p, Adams-B. 3p, Adams-M. 3p, Adams-B. 4p, Adams-M. 4p, Adams-B. 5p};
                \end{semilogyaxis}
            \end{tikzpicture}
            \caption{Sensibilidade do erro relativo máximo da aproximação da vazão de água acumulada com os métodos de Fetkovich, Runge-Kutta, Adams-Bashforth e Adams-Moulton.}
            \label{fig:testeAqSensMax}
        \end{figure}

        O código foi implementado em C++ e em um único arquivo. Pode ser encontrado em \href{https://github.com/TiagoCAAmorim/numerical-methods/blob/main/08_Adams/08_Adams.cpp}{https://github.com/Tiago CAAmorim/numerical-methods}.

    \section{Conclusão}

        Os métodos de Adams-Bashforth e Adams-Moulton mostraram melhor desempenho que os métodos de Runge-Kutta de ordem equivalente. Para o problema proposto de resolver o aquífero analítico proposto por Fetkovich, os métodos de passo múltiplo tiveram melhor desempenho que o método de Fetkovich. Entre os métodos implícitos e explícitos avaliados, o método explícito de maior ordem (Adams-Bashforth de 5 passos) foi que teve os melhores resultados.

    % \label{}

%% The Appendices part is started with the command \appendix;
%% appendix sections are then done as normal sections

\appendix

%% \section{}
%% \label{}

%% If you have bibdatabase file and want bibtex to generate the
%% bibitems, please use
%%

\bibliographystyle{elsarticle-num}
\bibliography{refs}

%% else use the following coding to input the bibitems directly in the
%% TeX file.

% \begin{thebibliography}{00}

%% \bibitem{label}
%% Text of bibliographic item

% \bibitem{}

% \end{thebibliography}

% \newpage
% \FloatBarrier
% \section{Código em C}

% O código de ambos métodos foi implementado em um único arquivo. O código é apresentado em duas partes neste documento para facilitar a leitura. O código pode ser encontrado em \href{https://github.com/TiagoCAAmorim/numerical-methods}{https://github.com/TiagoCAAmorim/numerical-methods}.

% \subsection{Método da Bissecção}
% \lstinputlisting[language=C, linerange={1-229}]{./02_newton_raphson.c}

% \subsection{Método de Newton-Raphson}
% \lstinputlisting[language=C, linerange={231-445}]{./02_newton_raphson.c}

% \subsection{Método da Mínima Curvatura}
% \lstinputlisting[language=C, linerange={448-958}]{./02_newton_raphson.c}

\end{document}
\endinput
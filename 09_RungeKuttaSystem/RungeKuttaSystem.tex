\documentclass[final,5p]{elsarticle}

% \documentclass[preprint,12pt]{elsarticle}

%% Use the option review to obtain double line spacing
%% \documentclass[authoryear,preprint,review,12pt]{elsarticle}

%% Use the options 1p,twocolumn; 3p; 3p,twocolumn; 5p; or 5p,twocolumn
%% for a journal layout:
% \documentclass[final,1p,times]{elsarticle}
%% \documentclass[final,1p,times,twocolumn]{elsarticle}
% \documentclass[final,3p,times]{elsarticle}
%% \documentclass[final,3p,times,twocolumn]{elsarticle}
% \documentclass[final,5p,times]{elsarticle}
%% \documentclass[final,5p,times,twocolumn]{elsarticle}
\usepackage[portuguese]{babel}

%% For including figures, graphicx.sty has been loaded in
%% elsarticle.cls. If you prefer to use the old commands
%% please give \usepackage{epsfig}

%% The amssymb package provides various useful mathematical symbols
\usepackage{amssymb}
\usepackage{amsmath}
\usepackage{multirow}

\usepackage{pgfplots}
\pgfplotsset{compat=1.18}
\usepgfplotslibrary{statistics}
\usepackage{pgfplotstable}

\usepackage{placeins}
\usepackage{hyperref}
\numberwithin{equation}{section}

\usepackage{algorithm}
\usepackage[noEnd=true, indLines=true]{algpseudocodex}
\algrenewcommand\algorithmicrequire{\textbf{Entrada:}}
\algrenewcommand\algorithmicwhile{\textbf{Enquanto}}
\algrenewcommand\algorithmicrepeat{\textbf{Repete}}
\algrenewcommand\algorithmicuntil{\textbf{Até}}
\algrenewcommand\algorithmicif{\textbf{Se}}
\algrenewcommand\algorithmicthen{\textbf{então}}
\algrenewcommand\algorithmicelse{\textbf{Caso contrário}}
\algrenewcommand\algorithmicensure{\textbf{Objetivo:}}
\algrenewcommand\algorithmicreturn{\textbf{Retorna:}}
\algrenewcommand\algorithmicdo{\textbf{faça}}
\algrenewcommand\algorithmicforall{\textbf{Para todos}}
\algnewcommand{\LineComment}[1]{\State \(\triangleright\) \textcolor{black!50}{\emph{#1}}}

% \usepackage[fleqn]{nccmath}
% \usepackage{multicol}


%=========== Gloabal Tikz settings
% \pgfplotsset{compat=newest}
% \usetikzlibrary{math}
% \pgfplotsset{
%     height = 10cm,
%     width = 10cm,
%     tick pos = left,
%     legend style={at={(0.98,0.30)}, anchor=east},
%     legend cell align=left,
%     }
%  \pgfkeys{
%     /pgf/number format/.cd,
%     fixed,
%     precision = 1,
%     set thousands separator = {}
% }

%% The amsthm package provides extended theorem environments
%% \usepackage{amsthm}

%% The lineno packages adds line numbers. Start line numbering with
%% \begin{linenumbers}, end it with \end{linenumbers}. Or switch it on
%% for the whole article with \linenumbers.
%% \usepackage{lineno}

\usepackage{listings}
\usepackage{xcolor}

\definecolor{codegreen}{rgb}{0,0.6,0}
\definecolor{codegray}{rgb}{0.5,0.5,0.5}
\definecolor{codepurple}{rgb}{0.58,0,0.82}
\definecolor{backcolour}{rgb}{0.98,0.98,0.98}

\lstdefinestyle{mystyle}{
    backgroundcolor=\color{backcolour},
    commentstyle=\color{codegreen},
    keywordstyle=\color{magenta},
    numberstyle=\tiny\color{codegray},
    stringstyle=\color{codepurple},
    basicstyle=\ttfamily\footnotesize,
    breakatwhitespace=false,
    breaklines=true,
    captionpos=b,
    keepspaces=true,
    numbers=left,
    numbersep=5pt,
    showspaces=false,
    showstringspaces=false,
    showtabs=false,
    tabsize=2
}

\lstset{style=mystyle}

% \journal{Nuclear Physics B}

\begin{document}

\begin{frontmatter}

%% Title, authors and addresses

%% use the tnoteref command within \title for footnotes;
%% use the tnotetext command for theassociated footnote;
%% use the fnref command within \author or \address for footnotes;
%% use the fntext command for theassociated footnote;
%% use the corref command within \author for corresponding author footnotes;
%% use the cortext command for theassociated footnote;
%% use the ead command for the email address,
%% and the form \ead[url] for the home page:
%% \title{Title\tnoteref{label1}}
%% \tnotetext[label1]{}
%% \author{Name\corref{cor1}\fnref{label2}}
%% \ead{email address}
%% \ead[url]{home page}
%% \fntext[label2]{}
%% \cortext[cor1]{}
%% \affiliation{organization={},
%%             addressline={},
%%             city={},
%%             postcode={},
%%             state={},
%%             country={}}
%% \fntext[label3]{}

\title{Uso de Métodos de Runge-Kutta para Resolver o Aquífero Analítico de Fetkovich\tnoteref{label_title}}
\tnotetext[label_title]{Relatório número 9 como parte dos requisitos da disciplina IM253: Métodos Numéricos para Fenômenos de Transporte.}

%% use optional labels to link authors explicitly to addresses:
%% \author[label1,label2]{}
%% \affiliation[label1]{organization={},
%%             addressline={},
%%             city={},
%%             postcode={},
%%             state={},
%%             country={}}
%%
%% \affiliation[label2]{organization={},
%%             addressline={},
%%             city={},
%%             postcode={},
%%             state={},
%%             country={}}

\author{Tiago C. A. Amorim\fnref{label_author}}
\tnotetext[label_author]{Atualmente cursando doutorado no Departamento de Engenharia de Petróleo da Faculdade de Engenharia Mecânica da UNICAMP (Campinas/SP, Brasil).}
\ead{t100675@dac.unicamp.br}
\affiliation[Tiago C. A. Amorim]{organization={Petrobras},%Department and Organization
addressline={Av. Henrique Valadares, 28},
city={Rio de Janeiro},
postcode={20231-030},
state={RJ},
country={Brasil}}

\begin{abstract}

    O Método de Fetkovich para prever o comportamento de um aquífero analítico ainda é útil mais de 50 anos após o seu desenvolvimento. Em avaliações anteriores, alguns métodos numérico conseguiram resultados de equivalente qualidade para um problema simples. Ao comparar o Método de Fetkovich com o Método de Runge-Kutta de quarta ordem para um problema de teste de produção, o Método de Fetkovich mostrou-se muito superior.

\end{abstract}


%%Graphical abstract
% \begin{graphicalabstract}
%\includegraphics{grabs}
% \end{graphicalabstract}

%%Research highlights
% \begin{highlights}
% \item Research highlight 1
% \item Research highlight 2
% \end{highlights}

\begin{keyword}
    Método de Fetkovich \sep Métodos de Runge-Kutta \sep Fluxo em Meio Poroso
%% keywords here, in the form: keyword \sep keyword

%% PACS codes here, in the form: \PACS code \sep code

%% MSC codes here, in the form: \MSC code \sep code
%% or \MSC[2008] code \sep code (2000 is the default)

\end{keyword}

\end{frontmatter}

%% \linenumbers

%% main text
\section{Introdução}

    Este trabalho se propõe a avaliar a adequação dos métodos de Runge-Kutta para resolver o modelo de aquífero analítico proposto por Fetkovich \cite{fetkovich1971simplified}. Em avaliações anteriores \cite{relatorioeuler}\cite{relatoriorungekutta}\cite{relatorioadams} foram testados diferentes métodos numéricos de resolução de equações diferencias. Nestas avaliações foi aplicada a forma \emph{original} dos métodos, ou seja, que resolve apenas uma equação diferencial de primeira ordem ($\frac{dy}{dt}=f(t,y)$). Esta limitação levou a uma restrição nos tipos de problemas que podem ser resolvidos.

    Neste relatório será avaliada a forma dos métodos de Runge-Kutta que resolvem sistemas de equações lineares. Com esta modificação a gama de problemas que podem ser resolvidas aumenta substancialmente. Foi avaliado um problema mais próximo de uma aplicação real. Foi modelado o problema de um poço que produz óleo durante certo tempo a uma vazão constante e que posteriormente é fechado, o que representa um teste de produção.

\section{Metodologia}

    \subsection{Métodos de Runge-Kutta}

    Os métodos de Runge-Kutta para resolver numericamente sistemas de equações diferenciais tem a mesma forma de suas respectivas versões para resolver uma equação diferencial. A diferença para o problema de resolver um sistema de equações diferenciais é que os parâmetros $k_i$ serão resolvidos para cada equação \cite{burden2016analise}.

    Dado um sistema de $m$ equações diferenciais na forma:

    \begin{align}
        \frac{dy_1}{dt} &= f_1(t, \overrightarrow{y}) \nonumber \\
        \frac{dy_2}{dt} &= f_2(t, \overrightarrow{y}) \nonumber \\
        \vdots \nonumber \\
        \frac{dy_m}{dt} &= f_m(t, \overrightarrow{y}) \label{eq:pvi}
    \end{align}
    \noindent com
    \begin{align}
        \overrightarrow{y} = \left\{ y_1, y_2, \ldots, y_m \right\} \nonumber
    \end{align}
    \noindent e condições iniciais
    \begin{align}
        y_1(t_0) = \alpha_1, \; y_1(t_0) = \alpha_2, \; \ldots, \; y_m(t_0) = \alpha_m \nonumber
    \end{align}

    Um Método de Runge-Kutta com $r$ parâmetros $k$, para resolver \ref{eq:pvi} com $w_{j,i} \approx y_i(t_j)$ em $n$ passos de tempo, tem a seguinte forma:

    \begin{align}
        w_{0,i} &= \alpha_i \nonumber \\
        k_{1,i} &= hf_j\left(t_j,\overrightarrow{w}_j\right) \nonumber \\
        k_{2,i} &= hf_j\left(t_j + \beta_2 h,\overrightarrow{w}_j + \gamma_{2,1} \overrightarrow{k}_1\right) \nonumber \\
        k_{3,i} &= hf_j\left(t_j + \beta_3 h,\overrightarrow{w}_j + \gamma_{3,1} \overrightarrow{k}_1 + \gamma_{3,2} \overrightarrow{k}_2\right) \nonumber \\
        \vdots \nonumber \\
        k_{r,i} &= hf_j\left(t_j + \beta_r h,\overrightarrow{w}_j + \sum_{s=1}^{r-1}\gamma_{r,s} \overrightarrow{k}_s\right) \nonumber \\
        \overrightarrow{w}_{j+1} &= \overrightarrow{w}_{j} + \sum_{s=1}^{r} \lambda_s \overrightarrow{k}_s \label{eq:RKgenerica}
    \end{align}
    \noindent com
    \begin{align}
        \overrightarrow{w}_j &= \left\{ w_{j,1}, w_{j,2}, \ldots, w_{j,m} \right\} \nonumber \\
        \overrightarrow{k}_s &= \left\{ k_{s,1}, k_{s,2}, \ldots, k_{s,m} \right\} \nonumber \\
        h &= \frac{t_n-t_0}{n} \nonumber \\
        i &= 1,\ldots,m \nonumber \\
        j &= 0,1,\ldots,n-1 \nonumber
    \end{align}

    Os parâmetros $\beta$, $\gamma$ e $\lambda$ dependem da escolha da versão do método de Runge-Kutta a ser utilizado. Usualmente são apresentados de forma tabular:

    \bigskip
    \renewcommand{\arraystretch}{1.2}
    \begin{tabular}{c|ccccc}
        0         &                &                &          &                  &    \\
        $\beta_2$ & $\gamma_{2,1}$ &                &          &                  &    \\
        $\beta_3$ & $\gamma_{3,1}$ & $\gamma_{3,2}$ &          &                  &    \\
        $\vdots$  & $\vdots$       &                & $\ddots$ &                  &    \\
        $\beta_r$ & $\gamma_{r,1}$ & $\gamma_{r,2}$ & $\cdots$ & $\gamma_{r,r-1}$ &    \\
        \hline
        & $\lambda_1$    & $\lambda_2$    & $\cdots$ & $\lambda_{r-1}$  & $\lambda_r$ \\
    \end{tabular}
    \bigskip

    Foram implementados métodos de Runge-Kutta de diferentes ordens (com as respectivas tabelas de parâmetros):

    \begin{itemize}
        \item Runge-Kutta de $1^a$ ordem: Método de Euler.
    \end{itemize}
    \hspace{2em}
    \begin{tabular}{c|c}
        0 &  \\
        \hline
          & 1    \\
    \end{tabular}
    \bigskip

    \begin{itemize}
        \item Runge-Kutta de $2^a$ ordem: Método do Ponto Médio.
    \end{itemize}
    \hspace{2em}
    \renewcommand{\arraystretch}{1.2}
    \begin{tabular}{c|cc}
        0            &               &   \\
       $\frac{1}{2}$ & $\frac{1}{2}$ &   \\
       \hline
                     & 0             & 1 \\
    \end{tabular}
    \bigskip

    \begin{itemize}
        \item Runge-Kutta de $3^a$ ordem: Método de Heun.
    \end{itemize}
    \hspace{2em}
    \renewcommand{\arraystretch}{1.2}
    \begin{tabular}{c|ccc}
        0             &               &               & \\
        $\frac{1}{3}$ & $\frac{1}{3}$ &               & \\
        $\frac{2}{3}$ & 0             & $\frac{1}{3}$ & \\
        \hline
                      & $\frac{1}{4}$ & 0             & $\frac{3}{4}$  \\
    \end{tabular}
    \bigskip

    \begin{itemize}
        \item Runge-Kutta de $4^a$ ordem: Versão \emph{clássica}.
    \end{itemize}
    \hspace{2em}
    \renewcommand{\arraystretch}{1.2}
    \begin{tabular}{c|cccc}
        0             &               &               &               & \\
        $\frac{1}{2}$ & $\frac{1}{2}$ &               &               & \\
        $\frac{1}{2}$ & 0             & $\frac{1}{2}$ &               & \\
        1             & 0             & 0             & 1             & \\
        \hline
                      & $\frac{1}{6}$ & $\frac{1}{3}$ & $\frac{1}{3}$ & $\frac{1}{6}$ \\
    \end{tabular}
    \bigskip

    \begin{itemize}
        \item Runge-Kutta de $5^a$ ordem: Primeira fórmula do Método de Runge-Kutta-Fehlberg.
    \end{itemize}
    \hspace{2em}
    \renewcommand{\arraystretch}{1.2}
    \begin{tabular}{c|cccccc}
        0               &                     &                      &                      &                     &                   &  \\
        $\frac{1}{4}$   & $\frac{1}{4}$       &                      &                      &                     &                   &  \\
        $\frac{3}{8}$   & $\frac{3}{32}$      & $\frac{9}{32}$       &                      &                     &                   &  \\
        $\frac{12}{13}$ & $\frac{1932}{2197}$ & -$\frac{7200}{2197}$ & $\frac{7296}{2197}$  &                     &                   &  \\
        1               & $\frac{439}{216}$   & -8                   & $\frac{3680}{513}$   & -$\frac{845}{4104}$ &                   &  \\
        $\frac{1}{2}$   & -$\frac{8}{27}$     & 2                    & -$\frac{3544}{2565}$ & $\frac{1859}{4104}$ & -$\frac{11}{40}$  &  \\
        \hline
                        & $\frac{16}{135}$    & 0                    & $\frac{6656}{12825}$ & $\frac{28561}{56430}$ & -$\frac{9}{50}$ & $\frac{2}{55}$ \\
    \end{tabular}
    \bigskip

    \subsection{Aquífero de Fetkovich como Sistema de Equações Diferenciais}

        Uma maior discussão sobre o modelo de aquífero proposto por Fetkovich e o método de resolução que leva seu nome é apresentada em \cite{relatorioeuler}. De forma resumida, o comportamento do modelo de aquífero tem a seguinte forma:

        \begin{align}
            \frac{d^2W_e}{dt^2} &= - \frac{J p_{i,aq}}{W_{e,max}} \frac{dW_e}{dt} - J \frac{dp_{res}}{dt} \label{eq:dwe2dt2}
        \end{align}

        Definindo $\frac{dW_e}{dt}$ como $y_1(t)$ e $W_e$ como $y_2(t)$, podemos reescrever \ref{eq:dwe2dt2} como um sistema de duas equações diferenciais:

        \begin{align}
            \frac{dy_1}{dt} &= - \frac{J p_{i,aq}}{W_{e,max}} y_1 - J \frac{dp_{res}}{dt} \nonumber \\
            \frac{dy_2}{dt} &= y_1 \label{eq:sistemadwe}
        \end{align}
        \noindent com condições iniciais:
        \begin{align}
            y_1(t_0) &= J (p_{i,aq} - p_{i,res}) \nonumber \\
            y_2(t_0) &= 0 \nonumber
        \end{align}


    \subsection{Modelo de Reservatório Proposto}

        O termo $\frac{dp_{res}}{dt}$ será função do modelo de reservatório proposto. Para um reservatório em que o equilíbrio hidrostático é alcançado \emph{instantaneamente}\footnote{Este comportamento é equivalente a um reservatório com altas permeabilidades, bem conectado e com óleo pouco compressível.} é possível demonstrar que\footnote{Maiores detalhes em \cite{relatorioeuler}}:

        \begin{align}
            p_{res} = \frac{N Bo_b (1+c_{o,b} p_b) + W_{res} - Pv_i (1 - c_r p_{i,res})}
            {N Bo_b c_{o,b} + Pv_i c_r }  \label{eq:pres}
        \end{align}
        \noindent onde
        \begin{align}
            N &= N_i - Np \nonumber \\
            W_{res} &= W_{i,res} + W_e \nonumber
        \end{align}

        Derivando \ref{eq:pres} no tempo:

        \begin{align}
            \frac{dp_{res}}{dt} =& \frac{\frac{dNp}{dt} Bo_b [c_{o,b}(p_{res}-p_b) - 1] + \frac{dW_e}{dt}}{(N_i - Np) Bo_b c_{o,b} + Pv_i c_r} \label{eq:dpresdt}
        \end{align}

        Estas equações são inseridas em \ref{eq:sistemadwe} para resolver o problema proposto. No caso de um teste de produção, em que a vazão de óleo ($\frac{dNp}{dt}$) e a produção acumulada ($Np$) são conhecidas, serão necessárias apenas duas equações diferenciais. Para um problema em que a vazão de óleo é função da pressão no reservatório, duas novas equações diferenciais são adicionadas, governando $\frac{dNp}{dt}$ e $Np$.

\section{Implementação} \label{sec:implementacao}

        Todo o código utilizado nesta análise foi desenvolvido em C++. Foram criados objetos próprios para cada elemento integrante do problema proposto:

        \begin{description}
            \item[IVPSystem] Classe que define um problema de valor inicial na forma \ref{eq:pvi}.
            \begin{itemize}
                \item O usuário precisa especificar $f_i(t,\overrightarrow{y})$, $a$ (tempo inicial), $b$ (tempo final), $n$ (número de passos de tempo) e $y_i(a)$ (valores iniciais).
                \item O usuário também pode especificar as soluções exatas ($y_i=f_i(t)$), para calcular o erro de aproximação.
            \end{itemize}
            \item[Fetkovich] Classe que resolve o comportamento de um aquífero como proposto por Fetkovich.
            \begin{itemize}
                \item O usuário precisa definir as características do aquífero e prover uma função que retorne a pressão na interface do aquífero com o reservatório. Esta função depende do tempo e do influxo acumulado de água do aquífero para o reservatório ($W_e$).
                \item Uma modificação foi feita na Implementação do Método de Fetkovich com relação aos testes apresentados nos relatórios anteriores.
                \begin{itemize}
                    \item Como existe um termo implícito a ser resolvido no método, anteriormente foi admitido realizar um cálculo com até 20 iterações.
                    \item Aplicando a mesma filosofia dos métodos preditor-corretor, este limite agora é de apenas uma iteração. E nos testes realizados os resultados foram muito parecidos.
                    \item Desta forma o Método de Fetkovich agora tem apenas duas avaliações da função de pressão a cada passo de tempo.
                \end{itemize}
            \end{itemize}
        \end{description}

\section{Resultados}

        Nesta avaliação o foco foi no método de Runge-Kutta de quarta ordem. Foram utilizados os mesmos três problemas de valor inicial dos relatórios anteriores. As aproximações de $y(t)$ foram, como esperado, iguais às da aplicação do método de Runge-Kutta com uma equação diferencial. A diferença foi a estimativa da integral de $y(t)$. Anteriormente este cálculo foi feito com os métodos de Simpson e do Trapézio compostos\footnote{O expediente utilizado para calcular esta integral numérica está descrito em \cite{relatoriorungekutta}}. Agora este cálculo foi inserido no método de Runge-Kutta.

        \begin{enumerate}
            \item $y' = y - t^2 + 1, \quad 0 \leq t \leq  2, \quad y(0) = 0.5$ \label{item:pvi1}
            \item $y' = -2y + 3 e^t, \quad 0 \leq t \leq  2, \quad y(0) = 3.0$ \label{item:pvi2}
            \item $y' = 4 cos(t) - 8 sin(t) + 2 y, \quad 0 \leq t \leq  2, \quad y(0) = 3.0$ \label{item:pvi3}
        \end{enumerate}

        Os resultados exatos de cada problema são, respectivamente:

        \begin{enumerate}
            \item $y = (t+1)^2 - 0.5 e^t$
            \item $y = 2 e^{-2 t} + e^t$
            \item $y = 4 sin(t) + 3 e^{2 t}$
        \end{enumerate}

        Observa-se nas figuras \ref{fig:teste1}, \ref{fig:teste2} e \ref{fig:teste3} que os resultados da integral de $y(t)$ incluída nos cálculos de Runge-Kutta tem um comportamento mais suave, sem as oscilações da integração numérico com Simpson e Trapézio composto.

        \begin{figure}[hbt!]
            \begin{tikzpicture}
                \begin{semilogyaxis}[
                    grid=both,
                    xlabel = {$t$},
                    ylabel = {$|w_i-y(t)|/|y(t)|$},
                    legend style={at={(0.90,0.85)}, anchor=east, font=\footnotesize},
                    ]
                    \addplot[color=blue, solid, smooth, thick] table [x=t, y=S_error] {test1_rungekutta4_intN.txt};
                    \addplot[color=black, solid, smooth, thick] table [x=t, y=y2_err] {test1_rungekutta4_intRK.txt};
                    \legend{Simpson+Trapézio, Runge-Kutta};
                \end{semilogyaxis}
            \end{tikzpicture}
            \caption{Erro da aproximação da acumulada de $y(t)$ com o método de Runge-Kutta de quarta ordem e por integração numérica com Simpson e Trapézio Composto, do PVI número \ref{item:pvi1}.}
            \label{fig:teste1}
        \end{figure}

        \begin{figure}[hbt!]
            \begin{tikzpicture}
                \begin{semilogyaxis}[
                    grid=both,
                    xlabel = {$t$},
                    ylabel = {$|w_i-y(t)|/|y(t)|$},
                    legend style={at={(0.90,0.85)}, anchor=east, font=\footnotesize},
                    ]
                    \addplot[color=blue, solid, smooth, thick] table [x=t, y=S_error] {test2_rungekutta4_intN.txt};
                    \addplot[color=black, solid, smooth, thick] table [x=t, y=y2_err] {test2_rungekutta4_intRK.txt};
                    \legend{Simpson+Trapézio, Runge-Kutta};
                \end{semilogyaxis}
            \end{tikzpicture}
            \caption{Erro da aproximação da acumulada de $y(t)$ com o método de Runge-Kutta de quarta ordem e por integração numérica com Simpson e Trapézio Composto, do PVI número \ref{item:pvi2}.}
            \label{fig:teste2}
        \end{figure}

        \begin{figure}[hbt!]
            \begin{tikzpicture}
                \begin{semilogyaxis}[
                    grid=both,
                    xlabel = {$t$},
                    ylabel = {$|w_i-y(t)|/|y(t)|$},
                    legend style={at={(0.90,0.85)}, anchor=east, font=\footnotesize},
                    ]
                    \addplot[color=blue, solid, smooth, thick] table [x=t, y=S_error] {test3_rungekutta4_intN.txt};
                    \addplot[color=black, solid, smooth, thick] table [x=t, y=y2_err] {test3_rungekutta4_intRK.txt};
                    \legend{Simpson+Trapézio, Runge-Kutta};
                \end{semilogyaxis}
            \end{tikzpicture}
            \caption{Erro da aproximação da acumulada de $y(t)$ com o método de Runge-Kutta de quarta ordem e por integração numérica com Simpson e Trapézio Composto, do PVI número \ref{item:pvi3}.}
            \label{fig:teste3}
        \end{figure}

        O problema proposto a ser resolvido pelo Método de Fetkovich e com o Método de Runge-Kutta de quarta ordem é o de um aquífero inicialmente em equilíbrio com o reservatório ($p_{i,aq} = p_{i,res}$). É feito um teste de produção de 30 dias, com o poço produzindo $500 m^3/d$. Em seguida o poço é fechado por um longo tempo. Este problema é numericamente mais complicado que o avaliado nos relatórios anteriores, pois há uma mudança brusca de comportamento no momento em que o poço fecha. Não há solução analítica.

        Os parâmetros do aquífero e do reservatório seguem os mesmos valores apresentados em \cite{relatorioeuler}. A exceção é $p_{i,aq}$, que, como comentado acima, agora tem o mesmo valor que a pressão inicial do reservatório.

        Os testes foram realizados de modo que o número de avaliações da função de pressão fosse a mesma entre os métodos, ou seja, o Método de Fetkovich utilizou passos de tempo duas vezes menores que o de Runge-Kutta. Observa-se nas figuras \ref{fig:aqqw} e \ref{fig:we} que as respostas foram próximas, mas diferentes. A maior diferença aparece na estimativa de volume de água que passou do aquífero para o reservatório.

        \begin{figure}[hbt!]
            \begin{tikzpicture}
                \begin{axis}[
                    grid=both,
                    xlabel = {$t$},
                    ylabel = {$Qw [m^3/d]$},
                    legend style={at={(0.90,0.85)}, anchor=east, font=\footnotesize},
                    ]
                    \addplot[color=blue, solid, smooth, thick] table [x=Time, y=Wat.Flow] {aq1_fetkovich.txt};
                    \addplot[color=black, solid, smooth, thick] table [x=t, y=Qw_appr] {aq1_rungekutta4.txt};
                    \legend{Fetkovich, Runge-Kutta};
                \end{axis}
            \end{tikzpicture}
            \caption{Vazão de água com o Método de Fetkovich e com o Método de Runge-Kutta de quarta ordem.}
            \label{fig:aqqw}
        \end{figure}

        \begin{figure}[hbt!]
            \begin{tikzpicture}
                \begin{axis}[
                    grid=both,
                    xlabel = {$t$},
                    ylabel = {$We [m^3]$},
                    legend style={at={(0.90,0.35)}, anchor=east, font=\footnotesize},
                    ]
                    \addplot[color=blue, solid, smooth, thick] table [x=Time, y=Cumulative] {aq1_fetkovich.txt};
                    \addplot[color=black, solid, smooth, thick] table [x=t, y=We_appr] {aq1_rungekutta4.txt};
                    \legend{Fetkovich, Runge-Kutta};
                \end{axis}
            \end{tikzpicture}
            \caption{Acumulada de água com o Método de Fetkovich e com o Método de Runge-Kutta de quarta ordem.}
            \label{fig:we}
        \end{figure}

        Para avaliar a qualidade das respostas do Método de Runge-Kutta de quarta ordem para o problema proposto, foi feita uma estimativa com um Método de Runge-Kutta de quinta ordem com 100 vezes mais passos de tempo. A figura \ref{fig:testeerro} compara o erro de aproximação real (comparando com a resposta exata) com o erro de aproximação estimado (comparando com a resposta do Runge-Kutta de quinta ordem). Observa-se que a estimativa de erro está na mesma \emph{ordem de grandeza} do erro real.

        \begin{figure}[hbt!]
            \begin{tikzpicture}
                \begin{semilogyaxis}[
                    grid=both,
                    xlabel = {$t$},
                    ylabel = {$|w_i-y(t)|/|y(t)|$},
                    legend style={at={(0.50,0.95)}, anchor=east, font=\footnotesize},
                    ]
                    \addplot[color=blue, solid, smooth, thick] table [x=t, y=y2_err] {test1_rungekutta4.txt};
                    \addplot[color=blue, dashed, smooth, thick] table [x=t, y=y2_err] {test1_rungekutta4Est.txt};
                    \addplot[color=black, solid, smooth, thick] table [x=t, y=y2_err] {test2_rungekutta4.txt};
                    \addplot[color=black, dashed, smooth, thick] table [x=t, y=y2_err] {test2_rungekutta4Est.txt};
                    \addplot[color=red, solid, smooth, thick] table [x=t, y=y2_err] {test3_rungekutta4.txt};
                    \addplot[color=red, dashed, smooth, thick] table [x=t, y=y2_err] {test3_rungekutta4Est.txt};
                    \legend{PVI 1: exato, PVI 1: estimado, PVI 2: exato, PVI 2: estimado, PVI 3: exato, PVI 3: estimado};
                \end{semilogyaxis}
            \end{tikzpicture}
            \caption{Erro real e estimado da aproximação da acumulada de $y(t)$ com o método de Runge-Kutta de quarta ordem para os três PVI de teste.}
            \label{fig:testeerro}
        \end{figure}

        Ao comparar as resposta de Fetkovich e Runge-Kutta de quarta ordem com os resultados de aplicar o método de Runge-Kutta de quinta ordem com um maior número de passos de tempo (figura \ref{fig:weestimado}) fica claro que o Método de Fetkovich tem resultados melhores.

        \begin{figure}[hbt!]
            \begin{tikzpicture}
                \begin{axis}[
                    grid=both,
                    xlabel = {$t$},
                    ylabel = {$We [m^3]$},
                    legend style={at={(0.90,0.35)}, anchor=east, font=\footnotesize},
                    ]
                    \addplot[color=blue, solid, smooth, thick] table [x=Time, y=Cumulative] {aq1_fetkovich.txt};
                    \addplot[color=black, solid, smooth, thick] table [x=t, y=We_appr] {aq1_rungekutta4.txt};
                    \addplot[color=red, dashed, smooth, thick] table [x=t, y=We_appr] {aq1_rungekutta5Best.txt};
                    \legend{Fetkovich, Runge-Kutta $4^a$, Runge-Kutta $5^a$};
                \end{axis}
            \end{tikzpicture}
            \caption{Acumulada de água com o Método de Fetkovich e com o Métodos de Runge-Kutta de quarta e quinta ordem.}
            \label{fig:weestimado}
        \end{figure}

        O código foi implementado em C++ e em um único arquivo. Pode ser encontrado em \href{https://github.com/TiagoCAAmorim/numerical-methods/blob/main/09_RungeKuttaSystem/09_RungeKuttaSystem.cpp}{https://github.com/Tiago CAAmorim/numerical-methods}.

    \section{Conclusão}

        Foi possível testar problemas mais desafiadores ao passar de métodos de aproximação numérica de uma equação diferencial de primeira ordem para métodos que resolvem um sistema de equações. Para o problema proposto o Método de Runge-Kutta teve desempenho pior que o do Método de Fetkovich.

    % \label{}

%% The Appendices part is started with the command \appendix;
%% appendix sections are then done as normal sections

\appendix

\section{Lista de Variáveis}

\begin{description}
    \item[$Bo$:]Fator volume de formação do óleo no reservatório ($m^3/m^3$).
    \item[$Bo_b$:]Fator volume de formação do óleo no reservatório na pressão de bolha ($m^3/m^3$).
    \item[$Bw$:]Fator volume de formação da água no reservatório ($m^3/m^3$).
    \item[$c_r$:]Compressibilidade do volume poroso ($1/bar$).
    \item[$c_{o,b}$:]Compressibilidade do óleo na pressão de bolha ($1/bar$).
    \item[$c_{aq}$:]Compressibilidade total do aquífero ($1/bar$).
    \item[$J$:]Índice de produtividade do aquífero ($m^3/d/bar$).
    \item[$N$:]Volume de óleo no reservatório, medido em condições padrão ($m^3$).
    \item[$N_i$:]Volume de óleo no reservatório inicial, medido em condições padrão ($m^3$).
    \item[$Np$:]Volume de óleo produzido, medido em condições padrão ($m^3$).
    \item[$p_{aq}$:]Pressão média do aquífero ($bar$).
    \item[$p_{i,aq}$:]Pressão inicial do aquífero ($bar$).
    \item[$p_b$:]Pressão de bolha do óleo ($bar$).
    \item[$p_{res}$:]Pressão na interface entre o aquífero e o reservatório ($bar$).
    \item[$p_{i,res}$:]Pressão inicial na interface entre o aquífero e o reservatório ($bar$).
    \item[$Pv$:]Volume poroso no reservatório ($m^3$).
    \item[$Pv_i$:]Volume poroso no reservatório na pressão inicial ($m^3$).
    \item[$Q_o$ ou $\frac{dNp}{dt}$:]Vazão de óleo produzido ($m^3/d$).
    \item[$Q_w$ ou $\frac{dW_e}{dt}$:]Vazão de água do aquífero para o reservatório ($m^3/d$).
    \item[$\Delta t_j$:]Diferença entre os tempos $t_{j-1}$ e $t_j$ ($d$).
    \item[$W_e$:]Volume de água acumulado do aquífero para o reservatório ($m^3$).
    \item[$W_{e,max}$:]Máximo influxo de água \emph{possível} do aquífero para o reservatório ($m^3$).
    \item[$(\Delta W_e)_j$:]Influxo de água entre os tempos $t_{j-1}$ e $t_j$ ($m^3$).
    \item[$W_{i,aq}$:]Volume inicial do aquífero ($m^3$).
    \item[$W_{res}$:]Volume de água no reservatório ($m^3$).
    \item[$W_{i,res}$:]Volume inicial de água no reservatório ($m^3$).
\end{description}

%% \section{}
%% \label{}

%% If you have bibdatabase file and want bibtex to generate the
%% bibitems, please use
%%

\bibliographystyle{elsarticle-num}
\bibliography{refs}

%% else use the following coding to input the bibitems directly in the
%% TeX file.

% \begin{thebibliography}{00}

%% \bibitem{label}
%% Text of bibliographic item

% \bibitem{}

% \end{thebibliography}

% \newpage
% \FloatBarrier
% \section{Código em C}

% O código de ambos métodos foi implementado em um único arquivo. O código é apresentado em duas partes neste documento para facilitar a leitura. O código pode ser encontrado em \href{https://github.com/TiagoCAAmorim/numerical-methods}{https://github.com/TiagoCAAmorim/numerical-methods}.

% \subsection{Método da Bissecção}
% \lstinputlisting[language=C, linerange={1-229}]{./02_newton_raphson.c}

% \subsection{Método de Newton-Raphson}
% \lstinputlisting[language=C, linerange={231-445}]{./02_newton_raphson.c}

% \subsection{Método da Mínima Curvatura}
% \lstinputlisting[language=C, linerange={448-958}]{./02_newton_raphson.c}

\end{document}
\endinput
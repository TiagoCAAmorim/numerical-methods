\documentclass[final,5p]{elsarticle}

% \documentclass[preprint,12pt]{elsarticle}

%% Use the option review to obtain double line spacing
%% \documentclass[authoryear,preprint,review,12pt]{elsarticle}

%% Use the options 1p,twocolumn; 3p; 3p,twocolumn; 5p; or 5p,twocolumn
%% for a journal layout:
% \documentclass[final,1p,times]{elsarticle}
%% \documentclass[final,1p,times,twocolumn]{elsarticle}
% \documentclass[final,3p,times]{elsarticle}
%% \documentclass[final,3p,times,twocolumn]{elsarticle}
% \documentclass[final,5p,times]{elsarticle}
%% \documentclass[final,5p,times,twocolumn]{elsarticle}
\usepackage[portuguese]{babel}

%% For including figures, graphicx.sty has been loaded in
%% elsarticle.cls. If you prefer to use the old commands
%% please give \usepackage{epsfig}

%% The amssymb package provides various useful mathematical symbols
\usepackage{amssymb}
\usepackage{amsmath}
\usepackage{multirow}

\usepackage{pgfplots}
\pgfplotsset{compat=1.18}
\usepgfplotslibrary{statistics}
\usepackage{pgfplotstable}

\usepackage{placeins}
\usepackage{hyperref}
\numberwithin{equation}{section}

\usepackage{algorithm}
\usepackage[noEnd=true, indLines=true]{algpseudocodex}
\algrenewcommand\algorithmicrequire{\textbf{Entrada:}}
\algrenewcommand\algorithmicwhile{\textbf{Enquanto}}
\algrenewcommand\algorithmicrepeat{\textbf{Repete}}
\algrenewcommand\algorithmicuntil{\textbf{Até}}
\algrenewcommand\algorithmicif{\textbf{Se}}
\algrenewcommand\algorithmicthen{\textbf{então}}
\algrenewcommand\algorithmicelse{\textbf{Caso contrário}}
\algrenewcommand\algorithmicensure{\textbf{Objetivo:}}
\algrenewcommand\algorithmicreturn{\textbf{Retorna:}}
\algrenewcommand\algorithmicdo{\textbf{faça}}
\algrenewcommand\algorithmicforall{\textbf{Para todos}}
\algnewcommand{\LineComment}[1]{\State \(\triangleright\) \textcolor{black!50}{\emph{#1}}}

% \usepackage[fleqn]{nccmath}
% \usepackage{multicol}


%=========== Gloabal Tikz settings
% \pgfplotsset{compat=newest}
% \usetikzlibrary{math}
% \pgfplotsset{
%     height = 10cm,
%     width = 10cm,
%     tick pos = left,
%     legend style={at={(0.98,0.30)}, anchor=east},
%     legend cell align=left,
%     }
%  \pgfkeys{
%     /pgf/number format/.cd,
%     fixed,
%     precision = 1,
%     set thousands separator = {}
% }

%% The amsthm package provides extended theorem environments
%% \usepackage{amsthm}

%% The lineno packages adds line numbers. Start line numbering with
%% \begin{linenumbers}, end it with \end{linenumbers}. Or switch it on
%% for the whole article with \linenumbers.
%% \usepackage{lineno}

\usepackage{listings}
\usepackage{xcolor}

\definecolor{codegreen}{rgb}{0,0.6,0}
\definecolor{codegray}{rgb}{0.5,0.5,0.5}
\definecolor{codepurple}{rgb}{0.58,0,0.82}
\definecolor{backcolour}{rgb}{0.98,0.98,0.98}

\lstdefinestyle{mystyle}{
    backgroundcolor=\color{backcolour},
    commentstyle=\color{codegreen},
    keywordstyle=\color{magenta},
    numberstyle=\tiny\color{codegray},
    stringstyle=\color{codepurple},
    basicstyle=\ttfamily\footnotesize,
    breakatwhitespace=false,
    breaklines=true,
    captionpos=b,
    keepspaces=true,
    numbers=left,
    numbersep=5pt,
    showspaces=false,
    showstringspaces=false,
    showtabs=false,
    tabsize=2
}

\lstset{style=mystyle}

% \journal{Nuclear Physics B}

\begin{document}

\begin{frontmatter}

%% Title, authors and addresses

%% use the tnoteref command within \title for footnotes;
%% use the tnotetext command for theassociated footnote;
%% use the fnref command within \author or \address for footnotes;
%% use the fntext command for theassociated footnote;
%% use the corref command within \author for corresponding author footnotes;
%% use the cortext command for theassociated footnote;
%% use the ead command for the email address,
%% and the form \ead[url] for the home page:
%% \title{Title\tnoteref{label1}}
%% \tnotetext[label1]{}
%% \author{Name\corref{cor1}\fnref{label2}}
%% \ead{email address}
%% \ead[url]{home page}
%% \fntext[label2]{}
%% \cortext[cor1]{}
%% \affiliation{organization={},
%%             addressline={},
%%             city={},
%%             postcode={},
%%             state={},
%%             country={}}
%% \fntext[label3]{}

\title{Aplicação do Método de Euler para Resolver o Comportamento de Aquíferos Analíticos\tnoteref{label_title}}
\tnotetext[label_title]{Relatório número 5 como parte dos requisitos da disciplina IM253: Métodos Numéricos para Fenômenos de Transporte.}

%% use optional labels to link authors explicitly to addresses:
%% \author[label1,label2]{}
%% \affiliation[label1]{organization={},
%%             addressline={},
%%             city={},
%%             postcode={},
%%             state={},
%%             country={}}
%%
%% \affiliation[label2]{organization={},
%%             addressline={},
%%             city={},
%%             postcode={},
%%             state={},
%%             country={}}

\author{Tiago C. A. Amorim\fnref{label_author}}
\tnotetext[label_author]{Atualmente cursando doutorado no Departamento de Engenharia de Petróleo da Faculdade de Engenharia Mecânica da UNICAMP (Campinas/SP, Brasil).}
\ead{t100675@dac.unicamp.br}
\affiliation[Tiago C. A. Amorim]{organization={Petrobras},%Department and Organization
addressline={Av. Henrique Valadares, 28},
city={Rio de Janeiro},
postcode={20231-030},
state={RJ},
country={Brasil}}

\begin{abstract}



\end{abstract}


%%Graphical abstract
% \begin{graphicalabstract}
%\includegraphics{grabs}
% \end{graphicalabstract}

%%Research highlights
% \begin{highlights}
% \item Research highlight 1
% \item Research highlight 2
% \end{highlights}

\begin{keyword}
    Método de Fetkovich \sep Método de Euler \sep Fluxo em Meio Poroso
%% keywords here, in the form: keyword \sep keyword

%% PACS codes here, in the form: \PACS code \sep code

%% MSC codes here, in the form: \MSC code \sep code
%% or \MSC[2008] code \sep code (2000 is the default)

\end{keyword}

\end{frontmatter}

%% \linenumbers

%% main text
\section{Introdução}

    % A simulação numérica tridimensional é uma das principais ferramentas no estudo da explotação de acumulações de hidrocarbonetos \cite{ReservoirSimulationErtekin}. Um simulador de fluxo em meio poroso é focado na resolução das variáveis de reservatório (pressão e saturações). Apesar de ser possível realizar simulações integradas de simuladores de fluxo em meio poroso com simuladores de fluxo multifásico \cite{10.2118/195477-MS}, o método mais utilizado para realizar o acoplamento das condições de fluxo de fundo de um poço com os limites operacionais das facilidades de produção são as tabelas de fluxo vertical multifásico, mais comumente conhecidas pela sigla do termo em inglês: VFP - \emph{vertical flow performance}\footnote{Também é comum encontrar o termo VLP - \emph{vertical lift performance}.}.

    % , desenvolveu um método analítico para descrever o influxo de um aquífero conectado ao um reservatório

\section{Metodologia}

    \subsection{Método de Fetkovich}

        Um dos modelos de aquífero analítico mais difundidos é o de Fetkovich \cite{schlumberger2009technical} \cite{computer2022cmg}. O método de Fetkovich se baseia em um modelo transiente, e em muitos casos se aproxima das respostas do mais sofisticado método de van Everdingen-Hurst \cite{VanEverdingen-Hurst}. A vantagem do método de Fetkovich é a simplicidade de sua aplicação, pois não depende de técnicas de superposição.

        O modelo de Fetkovich assume que a contribuição de um aquífero pode ser adequadamente modelada por um índice de produtividade, ou seja, que o influxo de água do aquífero para o reservatório é diretamente proporcional à diferença entre a pressão média do aquífero e a pressão na fronteira entre o reservatório e o aquífero \cite{fetkovich1971simplified}. O método negligencia quaisquer efeitos transientes no aquífero. O efeito desta premissa será maior em casos onde existe uma \emph{rápida} variação na pressão na interface entre o reservatório e o aquífero, fazendo com que os resultados do método de Fetkovich se desviem dos resultados mais rigorosos do método de van Everdingen-Hurst \cite{AHMED2019663}. Para problemas em que a variação desta pressão é mais gradual, o método de Fetkovich apresenta bons resultados.

        O método primeiramente assume um índice de produtividade constante para o aquífero:

        \begin{align}
            Q_w = \frac{dW_e}{dt} = J (p_{aq} - p_{res})& \label{eq:j}
        \end{align}

        A equação de balanço de massa assume que a compressibilidade do aquífero é constante, e que a depleção é proporcional à redução do volume de água:

        \begin{align}
            W_e = c_{aq} W_{i,aq} (p_{i,aq} - p_{aq})& \label{eq:compAquif}
        \end{align}

        A partir de \ref{eq:compAquif} é possível encontrar uma equação para a pressão média do aquífero:

        \begin{align}
            p_{aq} &= p_{i,aq} \left( 1 - \frac{W_e}{c_{aq} W_{i,aq} p_{i,aq}} \right) \nonumber \\
            &= p_{i,aq} \left( 1 - \frac{W_e}{W_{e,max}} \right) \label{eq:pAquif}
        \end{align}

        Diferenciando \ref{eq:pAquif} no tempo:

        \begin{align}
            \frac{dp_{aq}}{dt} &= - \frac{p_{i,aq}}{W_{e,max}} \frac{dW_e}{dt} \label{eq:dpaqdt}
        \end{align}

        Isolando $p_{aq}$ em \ref{eq:j} e diferenciando no tempo:

        \begin{align}
            \frac{dp_{aq}}{dt} &= \frac{1}{J} \frac{d^2W_e}{dt^2} + \frac{dp_{res}}{dt} \label{eq:dpaqdt2}
        \end{align}

        Igualando \ref{eq:dpaqdt} e \ref{eq:dpaqdt2}:

        \begin{align}
            \frac{d^2W_e}{dt^2} &= - \frac{J p_{i,aq}}{W_{e,max}} \frac{dW_e}{dt} - J \frac{dp_{res}}{dt} \label{eq:dwe2dt2}
        \end{align}

        Assumindo uma pressão constante na interface entre o aquífero e o reservatório ($\frac{dp_{res}}{dt} = 0$) e lembrando que $p_{aq}(t=0)=p_{i,aq}$, é possível resolver a EDO \ref{eq:dwe2dt2}:

        \begin{align}
            \frac{dW_e}{dt} &= J (p_{i,aq} - p_{res}) e^{\frac{J p_{i,aq}}{W_{e,max}} t} \label{eq:dwedt}
        \end{align}

        Integrando \ref{eq:dwedt} no tempo com $W_e(t=0)=0$:

        \begin{align}
            W_e &= \frac{W_{e,max}}{p_{i,aq}} (p_{i,aq} - p_{res}) \left( 1- e^{\frac{J p_{i,aq}}{W_{e,max}} t} \right) \label{eq:we}
        \end{align}

        Como a suposição de pressão constante na interface entre o aquífero e o reservatório é muito forte, Fetkovich propõe aplicar \ref{eq:we} de maneira incremental. Para um tempo $t_j$ com $j=1,2,\ldots,n$:

        \begin{align}
            (\Delta W_e)_j &= \frac{W_{e,max}}{p_{i,aq}} ((\overline{p}_{i,aq})_{j-1} - (\overline{p}_{res})_j) \left( 1- e^{\frac{J p_{i,aq}}{W_{e,max}} \Delta t_j} \right) \label{eq:deltawe}
        \end{align}

        \noindent
        onde

        \begin{align}
            (\overline{p}_{res})_j &= \frac{(p_{res})_j + (p_{res})_{j-1}}{2} \label{eq:presmedio}
        \end{align}

        \begin{align}
            (\overline{p}_{i,aq})_j &= p_{i,aq} \left( 1 - \frac{(\Delta W_e)_j}{W_{e,max}} \right) \label{eq:paqmedio}
        \end{align}

        Com $(p_{res})_0 = p_{i,res}$ e $(\overline{p}_{i,aq})_0 = p_{i,aq}$.

        Ao final do procedimento, o volume de água que passa do aquífero ao reservatório é a soma dos incrementos:

        \begin{align}
            W_e = \sum_{j = 1}^{n}  (\Delta W_e)_j \label{eq:wetotal}
        \end{align}

        O termo da pressão no aquífero é calculado no passo de tempo anterior. O único termo que não pode ser calculado diretamente é $(p_{res})_j$. Como este termo é normalmente função, entre outros, do influxo de água ($\sum_{j} (\Delta W_e)_j$), a cada passo de tempo é preciso resolver um problema do tipo $y = g(y)$.

    \subsection{Método de Euler}

        O objetivo do método de Euler é encontrar uma aproximação da solução de um problema de valor inicial bem posto (PVI):

        \begin{align}
            \frac{dy}{dt} = f(t,y), \quad a \leq t \leq b, \quad y(a) = \alpha \label{eq:pvi}
        \end{align}

        É possível usar o Teorema de Taylor para estimar o valor de $y(t_{j+1})$ a partir do valor de $y(t_j)$. Assumindo $y(t_{j+1}) - y(t_j) = h$ e com $\xi_j \; \epsilon \; (t_j, t_{j+1})$:

        \begin{align}
            y(t_{j+1}) = y(t_j) + h \left. \frac{dy}{dt} \right|_{t_j} + \frac{h^2}{2} \left. \frac{d^2y}{dt^2} \right|_{\xi_j} \label{eq:taylor}
        \end{align}

        Substituindo \ref{eq:pvi} em \ref{eq:taylor} e ignorando o resto, chega-se ao método de Euler. Fazendo $w_j \approx y(t_j)$, aplicando a discretização em $t$ em $n$ passos e a condição inicial:

        \begin{align}
            h &= \frac{b-a}{n} \nonumber \\
            t_j &= a + h j \nonumber \\
            w_0 &= \alpha \nonumber \\
            w_j &= w_{j-1} + h f(t_{j-1}, w_{j-1}) \label{eq:euler}
        \end{align}

        É possível mostrar que o limitante do erro do método de Euler é dado por\footnote{Prova completa em \cite{burden2016analise}}:

        \begin{align}
            \left\lvert y(t_j) - w_j \right\rvert &\leq \frac{h M}{2 L} \left[ e^{L(t_j-t_0)} -1 \right]  \label{eq:limitante}
        \end{align}

        \noindent
        onde

        \begin{align}
            \left\lvert \frac{\partial f}{\partial y} (t,y(t)) \right\rvert &\leq L \label{eq:L} \\
            \left\lvert \frac{d^2y}{dt^2} (t) \right\rvert = \left\lvert \frac{df}{dt} (t,y(t)) \right\rvert =& \nonumber \\
            \left\lvert \frac{\partial f}{\partial t} (t,y(t)) + \frac{\partial f}{\partial y} (t,y(t)) f(t,y(t)) \right\rvert &\leq M \label{eq:M}
        \end{align}

        A estimativa do limitante do erro pode ser feita mesmo sem conhecer explicitamente a função $y(t)$. Neste caso os valores de $L$ e $M$ serão estimados com os valores aproximados $w_j$.

        \subsection{Modelo de Fetkovich como PVI}

        Uma forma de resolver o modelo proposto por Fetkovich sem ter que assumir uma pressão constante na interface entre o aquífero e o reservatório é transformá-lo em um problema de valor inicial.

        Vamos assumir um modelo simples de reservatório em que é alcançado equilíbrio hidrostático \emph{instantaneamente}\footnote{Em um instante qualquer todo o reservatório estará na mesma pressão $p_{res}$.}. Por simplicidade é assumido que $Bw=1$.

        \begin{align}
            N Bo + W_{res} = Pv  \label{eq:reservatorio}
        \end{align}

        Serão assumidas correlações lineares do fator volume de formação do óleo com a pressão, e do volume poroso com a pressão.

        \begin{align}
            N Bo_b [1+c_{o,b}(p_b-p_{res})] &+ W_{res} = \nonumber \\
            & Pv_i [1 - c_r (p_{i,res} - p_{res})]  \label{eq:reservatorio2}
        \end{align}

        Isolando $p_{res}$ em \ref{eq:reservatorio2}:

        \begin{align}
            p_{res} = \frac{N Bo_b (1+c_{o,b} p_b) + W_{res} - Pv_i (1 - c_r p_{i,res})}
            {N Bo_b c_{o,b} + Pv_i c_r }  \label{eq:pres}
        \end{align}

        Derivando \ref{eq:pres} no tempo:

        \begin{align}
            \frac{dp_{res}}{dt} =& \frac{\frac{dN}{dt} Bo_b (1+c_{o,b} p_b) + \frac{dW_{res}}{dt}}
            {N Bo_b c_{o,b} + Pv_i c_r} + \nonumber \\
            & -p_{res}\frac{\frac{dN}{dt} Bo_b c_{o,b}}{N Bo_b c_{o,b} + Pv_i c_r} \label{eq:dpresdt}
        \end{align}

        Substituindo \ref{eq:dpresdt} em \ref{eq:dwe2dt2}, e assumindo $\frac{dN}{dt} = f(t,p_{res})$, encontra-se um PVI da forma:

        \begin{align}
            \frac{d^2y}{dt^2} = -\alpha \frac{dy}{dt} - f(t,y), \quad a \leq t \leq b, \quad y(a) = \alpha \label{eq:pvi2}
        \end{align}


        Para aplicar o método de Euler será preciso focar no caso específico em que não há produção de óleo ($\frac{dN}{dt} = 0$):

        \begin{align}
            \frac{d^2W_e}{dt^2} &= - J \left(\frac{p_{i,aq}}{W_{e,max}} + \frac{1}{N Bo_b c_{o,b} + Pv_i c_r} \right)  \frac{dW_e}{dt} \label{eq:dwe2dt2res}
        \end{align}

        Como a condição inicial de $p_{aq}(t=0)=p_{i,aq}$ se mantém, a EDO \ref{eq:dwe2dt2res} tem solução analítica análoga a \ref{eq:dwedt}:

        \begin{align}
            \frac{dW_e}{dt} &= J (p_{i,aq} - p_{res}) e^{J \left( \frac{p_{i,aq}}{W_{e,max}} + \frac{1}{N Bo_b c_{o,b} + Pv_i c_r} \right)  t} \label{eq:dwedtres}
        \end{align}

\section{Implementação}

        Todo o código utilizado nesta análise foi desenvolvido em C++. Foram criados objetos próprios para cada elemento integrante do problema proposto:

        \begin{description}
            \item[IVP] Classe que define um problema de valor inicial na forma \ref{eq:pvi}.
            \begin{itemize}
                \item O usuário precisa especificar $f(t,y)$, $a$ (tempo inicial), $b$ (tempo final), $n$ (número de passos de tempo) e $y(a)$.
                \item Opcionalmente o usuário pode prover $\frac{\partial f}{\partial y}$ e $\frac{df}{dt}$ para que seja estimado o limitante do erro de aproximação. O usuário também pode prover diretamente os valores de $L$ e $M$.
                \item O usuário também pode especificar a função exata, para calcular o erro de aproximação do método de Euler.
                \item Além do método de Euler \emph{tradicional}, foi implementada uma rotina que tenta melhorar as respostas utilizando o método de Aitken. Nesta última opção o método de Euler é feito com 3 diferentes passos de tempo: $n$, $2 n$ e $3 n$.
            \end{itemize}
            \item[Spline] Classe que constrói uma função interpoladora do tipo Spline natural a um conjunto de pontos $(x,y)$ \cite{relatoriosplinesnaturais}.
            \begin{itemize}
                \item Os valores das funções que definem $L$ e $M$ dependem de $y(t)$ (que é aproximado por $w$). Estas funções só podem ser avaliadas nos $n$ valores de $w_i$ que são calculados pelo método de Euler.
                \item Esta classe foi incluída no código para incrementar a busca pelos valores de $L$ e $M$. São criadas splines a partir dos valores calculados de cada função, de modo que um número maior de pontos podem ser avaliados.
            \end{itemize}
            \item[Fetkovich] Classe resolve o comportamento de um aquífero como proposto por Fetkovich.
            \begin{itemize}
                \item O usuário precisa definir as características do aquífero e prover uma função que retorne a pressão na interface do aquífero com o reservatório. Esta função depende do tempo e do influxo acumulado de água do aquífero para o reservatório ($W_e$).
            \end{itemize}
        \end{description}

        Para facilitar a estimativa de parâmetros de aquífero e reservatório foram implementadas algumas correlações \emph{clássicas}:

        \begin{itemize}
            \item Compressibilidade de rocha por Newman
            \item
        \end{itemize}


\section{Resultados}

        Para facilitar a análise da qualidade do código desenvolvido, foram criadas funções que realizam diversos testes onde a resposta exata é conhecida:




        O código foi implementado em C++ e em um único arquivo. Pode ser encontrado em \href{https://github.com/TiagoCAAmorim/numerical-methods/blob/main/06_Euler/06_Euler.cpp}{https://github.com/Tiago CAAmorim/numerical-methods}.

    \section{Conclusão}

        O desempenho de splines naturais para interpolar valores de pressão de fundo a partir de uma tabela de VFP foi, em geral, pior que o da interpolação linear. Ambos métodos apresentaram dificuldade em estimar valores na região de baixas vazões.

    % \label{}

%% The Appendices part is started with the command \appendix;
%% appendix sections are then done as normal sections

\appendix

\section{Lista de Variáveis}

\begin{description}
    \item[$Bo$:]Fator volume de formação do óleo no reservatório ($m^3/m^3$).
    \item[$Bo_b$:]Fator volume de formação do óleo no reservatório na pressão de bolha ($m^3/m^3$).
    \item[$Bw$:]Fator volume de formação da água no reservatório ($m^3/m^3$).
    \item[$c_r$:]Compressibilidade do volume poroso ($1/bar$).
    \item[$c_{o,b}$:]Compressibilidade do óleo na pressão de bolha ($1/bar$).
    \item[$c_{aq}$:]Compressibilidade total do aquífero ($1/bar$).
    \item[$J$:]Índice de produtividade do aquífero ($m^3/d/bar$).
    \item[$L$:]Constante de Lipschitz.
    \item[$M$:]Limitante da derivada 2$^a$.
    \item[$N$:]Volume de óleo no reservatório medido em condições padrão ($m^3$).
    \item[$p_{aq}$:]Pressão média do aquífero ($bar$).
    \item[$p_{i,aq}$:]Pressão inicial do aquífero ($bar$).
    \item[$(\overline{p}_{i,aq})_j$:]Pressão média do aquífero no tempo $t_j$ ($bar$).
    \item[$p_b$:]Pressão de bolha do óleo ($bar$).
    \item[$p_{res}$:]Pressão na interface entre o aquífero e o reservatório ($bar$).
    \item[$p_{i,res}$:]Pressão inicial na interface entre o aquífero e o reservatório ($bar$).
    \item[$(\overline{p}_{res})_j$:]Pressão média na interface entre o aquífero e o reservatório entre os tempos $t_{j-1}$ e $t_j$ ($bar$).
    \item[$Pv$:]Volume poroso no reservatório ($m^3$).
    \item[$Pv_i$:]Volume poroso no reservatório na pressão inicial ($m^3$).
    \item[$Q_w$ ou $\frac{dW_e}{dt}$:]Vazão de água do aquífero para o reservatório ($m^3/d$).
    \item[$\Delta t_j$:]Diferença entre os tempos $t_{j-1}$ e $t_j$ ($d$).
    \item[$W_e$:]Volume de água acumulado do aquífero para o reservatório ($m^3$).
    \item[$W_{e,max}$:]Máximo influxo de água \emph{possível}\footnote{Equivale ao influxo quando $p_{aq}=0$ em \ref{eq:compAquif}.} do aquífero para o reservatório ($m^3$).
    \item[$(\Delta W_e)_j$:]Influxo de água entre os tempos $t_{j-1}$ e $t_j$ ($m^3$).
    \item[$W_{i,aq}$:]Volume inicial do aquífero ($m^3$).
    \item[$W_{res}$:]Volume de água no reservatório ($m^3$).
\end{description}


%% \section{}
%% \label{}

%% If you have bibdatabase file and want bibtex to generate the
%% bibitems, please use
%%

\bibliographystyle{elsarticle-num}
\bibliography{refs}

%% else use the following coding to input the bibitems directly in the
%% TeX file.

% \begin{thebibliography}{00}

%% \bibitem{label}
%% Text of bibliographic item

% \bibitem{}

% \end{thebibliography}

% \newpage
% \FloatBarrier
% \section{Código em C}

% O código de ambos métodos foi implementado em um único arquivo. O código é apresentado em duas partes neste documento para facilitar a leitura. O código pode ser encontrado em \href{https://github.com/TiagoCAAmorim/numerical-methods}{https://github.com/TiagoCAAmorim/numerical-methods}.

% \subsection{Método da Bissecção}
% \lstinputlisting[language=C, linerange={1-229}]{./02_newton_raphson.c}

% \subsection{Método de Newton-Raphson}
% \lstinputlisting[language=C, linerange={231-445}]{./02_newton_raphson.c}

% \subsection{Método da Mínima Curvatura}
% \lstinputlisting[language=C, linerange={448-958}]{./02_newton_raphson.c}

\end{document}
\endinput
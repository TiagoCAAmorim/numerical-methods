\documentclass[final,5p]{elsarticle}

% \documentclass[preprint,12pt]{elsarticle}

%% Use the option review to obtain double line spacing
%% \documentclass[authoryear,preprint,review,12pt]{elsarticle}

%% Use the options 1p,twocolumn; 3p; 3p,twocolumn; 5p; or 5p,twocolumn
%% for a journal layout:
% \documentclass[final,1p,times]{elsarticle}
%% \documentclass[final,1p,times,twocolumn]{elsarticle}
% \documentclass[final,3p,times]{elsarticle}
%% \documentclass[final,3p,times,twocolumn]{elsarticle}
% \documentclass[final,5p,times]{elsarticle}
%% \documentclass[final,5p,times,twocolumn]{elsarticle}
\usepackage[portuguese]{babel}

%% For including figures, graphicx.sty has been loaded in
%% elsarticle.cls. If you prefer to use the old commands
%% please give \usepackage{epsfig}

%% The amssymb package provides various useful mathematical symbols
\usepackage{amssymb}
\usepackage{amsmath}
\usepackage{multirow}

\usepackage{pgfplots}
\pgfplotsset{compat=1.18}
\usepgfplotslibrary{statistics} 
\usepackage{pgfplotstable}

\usepackage{placeins}
\usepackage{hyperref}
\numberwithin{equation}{section}

\usepackage{algorithm}
\usepackage[noEnd=true, indLines=true]{algpseudocodex}
\algrenewcommand\algorithmicrequire{\textbf{Entrada:}}
\algrenewcommand\algorithmicwhile{\textbf{Enquanto}}
\algrenewcommand\algorithmicrepeat{\textbf{Repete}}
\algrenewcommand\algorithmicuntil{\textbf{Até}}
\algrenewcommand\algorithmicif{\textbf{Se}}
\algrenewcommand\algorithmicthen{\textbf{então}}
\algrenewcommand\algorithmicelse{\textbf{Caso contrário}}
\algrenewcommand\algorithmicensure{\textbf{Objetivo:}}
\algrenewcommand\algorithmicreturn{\textbf{Retorna:}}
\algrenewcommand\algorithmicdo{\textbf{faça}}
\algrenewcommand\algorithmicforall{\textbf{Para todos}}
\algnewcommand{\LineComment}[1]{\State \(\triangleright\) \textcolor{black!50}{\emph{#1}}}

% \usepackage[fleqn]{nccmath}
% \usepackage{multicol}


%=========== Gloabal Tikz settings
% \pgfplotsset{compat=newest}
% \usetikzlibrary{math}
% \pgfplotsset{
%     height = 10cm,
%     width = 10cm,
%     tick pos = left,
%     legend style={at={(0.98,0.30)}, anchor=east},
%     legend cell align=left,     
%     }
%  \pgfkeys{
%     /pgf/number format/.cd,
%     fixed,
%     precision = 1,
%     set thousands separator = {}
% }

%% The amsthm package provides extended theorem environments
%% \usepackage{amsthm}

%% The lineno packages adds line numbers. Start line numbering with
%% \begin{linenumbers}, end it with \end{linenumbers}. Or switch it on
%% for the whole article with \linenumbers.
%% \usepackage{lineno}

\usepackage{listings}
\usepackage{xcolor}

\definecolor{codegreen}{rgb}{0,0.6,0}
\definecolor{codegray}{rgb}{0.5,0.5,0.5}
\definecolor{codepurple}{rgb}{0.58,0,0.82}
\definecolor{backcolour}{rgb}{0.98,0.98,0.98}

\lstdefinestyle{mystyle}{
    backgroundcolor=\color{backcolour},   
    commentstyle=\color{codegreen},
    keywordstyle=\color{magenta},
    numberstyle=\tiny\color{codegray},
    stringstyle=\color{codepurple},
    basicstyle=\ttfamily\footnotesize,
    breakatwhitespace=false,         
    breaklines=true,                 
    captionpos=b,                    
    keepspaces=true,                 
    numbers=left,                    
    numbersep=5pt,                  
    showspaces=false,                
    showstringspaces=false,
    showtabs=false,                  
    tabsize=2
}

\lstset{style=mystyle}

% \journal{Nuclear Physics B}

\begin{document}

\begin{frontmatter}

%% Title, authors and addresses

%% use the tnoteref command within \title for footnotes;
%% use the tnotetext command for theassociated footnote;
%% use the fnref command within \author or \address for footnotes;
%% use the fntext command for theassociated footnote;
%% use the corref command within \author for corresponding author footnotes;
%% use the cortext command for theassociated footnote;
%% use the ead command for the email address,
%% and the form \ead[url] for the home page:
%% \title{Title\tnoteref{label1}}
%% \tnotetext[label1]{}
%% \author{Name\corref{cor1}\fnref{label2}}
%% \ead{email address}
%% \ead[url]{home page}
%% \fntext[label2]{}
%% \cortext[cor1]{}
%% \affiliation{organization={},
%%             addressline={},
%%             city={},
%%             postcode={},
%%             state={},
%%             country={}}
%% \fntext[label3]{}

\title{Aplicação de Splines Fixas para Interpolar Tabelas de Fluxo Vertical Multifásico\tnoteref{label_title}}
\tnotetext[label_title]{Relatório número 5 como parte dos requisitos da disciplina IM253: Métodos Numéricos para Fenômenos de Transporte.}

%% use optional labels to link authors explicitly to addresses:
%% \author[label1,label2]{}
%% \affiliation[label1]{organization={},
%%             addressline={},
%%             city={},
%%             postcode={},
%%             state={},
%%             country={}}
%%
%% \affiliation[label2]{organization={},
%%             addressline={},
%%             city={},
%%             postcode={},
%%             state={},
%%             country={}}

\author{Tiago C. A. Amorim\fnref{label_author}}
\tnotetext[label_author]{Atualmente cursando doutorado no Departamento de Engenharia de Petróleo da Faculdade de Engenharia Mecânica da UNICAMP (Campinas/SP, Brasil).}
\ead{t100675@dac.unicamp.br}
\affiliation[Tiago C. A. Amorim]{organization={Petrobras},%Department and Organization
addressline={Av. Henrique Valadares, 28}, 
city={Rio de Janeiro},
postcode={20231-030}, 
state={RJ},
country={Brasil}}

\begin{abstract}

    A construção de curvas de fluxo vertical multifásico para utilizar em simulação de fluxo em meios porosos pode ser uma tarefa computacionalmente custosa. Como uma tabela estática irá substituir um simulador de fluxo multifásico, é preciso um número adequado de parâmetros de entrada. A nuvem de pontos da tabela deve garantir que os valores a serem interpolados em uma simulação de fluxo em meios porosos sejam próximos dos valores reais.

    Tradicionalmente os simuladores de fluxo em meio poroso utilizam interpolação linear entre os pontos tabelados. Um primeiro estudo verificou que usar splines naturais para interpolar os valores tabelados não resulta em melhora das estimativas. Este segundo estudo utilizou os dados tabelados para estimar valores para as derivadas da função $P_{wf}=f(Q)$ no extremos do domínio, e assim poder construir splines fixadas. Os resultados mostraram que as splines fixadas conseguiram resultados melhores que os da interpolação com splines naturais ou linear.
    
\end{abstract}


%%Graphical abstract
% \begin{graphicalabstract}
%\includegraphics{grabs}
% \end{graphicalabstract}

%%Research highlights
% \begin{highlights}
% \item Research highlight 1
% \item Research highlight 2
% \end{highlights}

\begin{keyword}
    Splines \sep Simulação Numérica \sep Fluxo em Meio Poroso
%% keywords here, in the form: keyword \sep keyword

%% PACS codes here, in the form: \PACS code \sep code

%% MSC codes here, in the form: \MSC code \sep code
%% or \MSC[2008] code \sep code (2000 is the default)

\end{keyword}

\end{frontmatter}

%% \linenumbers

%% main text
\section{Introdução}

    As tabelas de fluxo vertical multifásico, mais comumente conhecidas pela sigla do termo em inglês: VFP - \emph{vertical flow performance}\footnote{Ou VLP - \emph{vertical lift performance}.}, são a abordagem mais usual para acoplar uma simulação de fluxo em meios porosos com os limites da unidade de produção.

    As tabelas de VFP associam diferentes variáveis de fluxo (vazões) com as pressões nos extremos do elemento modelado\footnote{Poço, pedaço de linha de produção, \emph{manifold} etc.}. As tabelas são geradas com um número limitado de pontos, e invariavelmente o simulador de fluxo terá que buscar valores para condições de fluxo que não estão na tabela. O mais comum é realizar interpolação linear entre os pontos da tabela \cite{computer2022cmg}\cite{schlumberger2009technical}. Um estudo anterior avaliou o uso de splines naturais para interpolar valores em tabelas de VFP, e concluiu que, para os testes realizados, a interpolação linear é melhor \cite{relatoriosplinesnaturais}. Este relatório revisita os mesmos exemplos e avalia a interpolação com splines fixas.

\section{Metodologia}

    \subsection{Tabelas de Fluxo Vertical Multifásico}

        Os simuladores de fluxo em meio poroso utilizam as tabelas de VFP para poder relacionar condições de fluxo de fundo e de superfície. O usual é que a tabela de VFP relacione pressão de fundo ($P_{wf}$\footnote{Ou BHP, que é \emph{Bottom Head Pressure}.}) com as demais variáveis associadas ao poço: 
        
        \begin{itemize}
            \item Pressão de cabeça: WHP\footnote{WHP é \emph{Well Head Pressure}.}.
            \item Vazão: óleo, gás, água ou líquido\footnote{Respectivamente $Q_o$, $Q_g$, $Q_w$,$Q_{liq}$.}.
            \item Fração de água: WCUT\footnote{WCUT é \emph{Water Cut} ou corte de água ($Q_w/Q_{liq}$)} ou RAO\footnote{RAO é razão água óleo ($Q_g/Q_o$).}
            \item Fração de gás: RGL\footnote{RGL é razão gás líquido ($Q_g/Q_{liq}$).} ou RGO\footnote{RGO é razão gás óleo ($Q_g/Q_o$).}..
            \item Vazão de injeção de \emph{gas-lift}.
        \end{itemize}

        No simulador de fluxo um poço terá sempre uma condição de produção especificada: vazão, pressão de fundo, pressão de cabeça ou vazão de grupo. Enquanto estiver em uma determinada condição de produção, os demais limitantes devem ser observados: pressão de fundo mínima, pressão de cabeça mínima, vazão de líquido máxima, vazão de óleo máxima, vazão de gás máxima\footnote{Para o caso de um produtor.}. Se algum limite não for satisfeito, o simulador irá mudar o controle do poço para o limite da condição violada e buscar uma nova solução para o problema. Este processo é realizado a cada passo de tempo da simulação, e a tabela de VFP será utilizada toda vez que for necessário calcular a pressão de cabeça do poço.

    \subsection{Splines}
    
        Dado um conjunto de $n+1$ pontos ($x_i,y_i \; para \; i=0,1,\ldots,n$), as splines são um método de interpolação que utiliza $n$ equações cúbicas por partes na forma:
        
        \begin{align*}
            S_j(x) = a_j + b_j (x-x_j) + c_j (x-x_j)^2 + d_j (x-x_j)^2& \\
            para \; x_j \le x \le x_{j+1}&
        \end{align*}
        
        Os termos das equações são definidos de forma a garantir que os valores da função interpoladora ($S(x)$) sejam exatos nos pontos dados ($S(x_i)=y_i$), e que exista continuidade da função interpoladora, e de suas derivadas primeira e segunda (mais detalhes em \cite{relatoriosplinesnaturais}). Duas condições adicionais precisam ser definidas para conseguir definir todos os termos da função interpoladora. Splines naturais tem derivada segunda nula nos extremos do domínio ($S''(x_0)=S''(x_n)=0$). Já as splines fixadas tem as derivadas primeiras nos extremos do domínio definidas: $S'(x_0)=f'(x_0)$ e $S'(x_n)=f'(x_n)$. O código desenvolvido para as splines fixas é o apresentado em \cite{burden2016analise}. Uma forma simplificada do algoritmo das splines fixas é apresentada no Algoritmo \ref{alg:splines}. 

        \begin{algorithm}
            \caption{Splines Fixas}\label{alg:splines}
            \begin{algorithmic}
                \Require $x_i,y_i \; para \; i=0,1,\ldots,n; \; y'_0=f'(x_0); \; y'_n=f'(x_n)$
                \ForAll{$i \in \{0, \dots, n\}$}
                    \State $a_i \gets y_i$
                \EndFor
                \ForAll{$i \in \{0, \dots, n-1\}$}
                    \State $h_i \gets x_{i+1} - x_i$
                \EndFor

                \State $\alpha_0 \gets \frac{3}{h_0}(a_1-a_0) - 3 y'_0$
                \State $\alpha_n \gets 3 y'_n - \frac{3}{h_{n-1}}(a_n-a_{n-1})$

                \ForAll{$i \in \{1, \dots, n-1\}$}
                    \State $\alpha_i \gets \frac{3}{h_i} (a_{i+1}-a_i) - \frac{3}{h_{i-1}} (a_i-a_{i-1})$
                \EndFor
                
                \State $r_0 \gets 2 h_0$
                \State $m_0 \gets 0.5$
                \State $z_0 \gets \frac{\alpha_0}{r_0}$
                
                \ForAll{$i \in \{1, \dots, n-1\}$}
                    \State $r_i \gets 2 (x_{i+1} - x_{i-1}) - h_{i-1} m_{i-1}$
                    \State $m_i \gets \frac{h_i}{r_i}$
                    \State $z_i \gets \frac{\alpha_i - h_{i-1} z_{i-1}}{r_i}$
                \EndFor
                
                \State $r_n \gets h_{n-1} (2 - m_{n-1})$
                \State $z_n \gets \frac{\alpha_n - h_{n-1} z_{n-1}}{r_n}$
                \State $c_n \gets z_n$
                
                \ForAll{$i \in \{n-1, \dots, 0\}$}
                    \State $c_i \gets z_i - m_i c_{i+1}$
                    \State $b_i \gets \frac{a_{i+1} - a_i}{h_i} - h_i \frac{c_{i+1} + 2 c_i}{3}$
                    \State $d_i \gets \frac{c_{i-1} - c_i}{3 h_i}$
                \EndFor
                \State \Return $a_i,b_i,c_i,d_i \; para \; i=0,1,\ldots,n-1$
            \end{algorithmic}
        \end{algorithm}
    
    \section{Resultados}
    
        Para facilitar a análise da qualidade do código desenvolvido, foram criadas funções que realizam diversos testes onde a resposta exata é conhecida:

        \begin{description}
            \item[tests\textunderscore splines()] Testa as implementações de splines naturais e splines fixadas em dois exemplos: conjunto de 3 pontos e aproximação da integral da função exponencial (exemplos 1 a 4 do capítulo 3.5 de \cite{burden2016analise}).
            
            \item[tests\textunderscore vfp\textunderscore interpolation()] Compara o uso de interpolação linear e interpolação com splines naturais e fixas de pontos de uma tabela de VFP.
        \end{description}

        Os testes que utilizam os exemplos de \cite{burden2016analise} tiveram resultados iguais aos do livro. O exemplo mais interessante é o da função exponencial, que é aproximada por splines nos pontos $x \; \epsilon \; \{0,1, 2, 3\}$ (Figura \ref{fig:exp}). O exemplo demonstra que a aproximação com spline fixada, impondo as derivadas da função original no primeiro ponto e no último ponto, melhora a estimativa da integral no intervalo aproximado por splines (Figura \ref{fig:intexp}).
        
        \begin{figure}[hbt!] 
            \begin{tikzpicture}
                \begin{axis}[
                    grid=both,
                    xlabel = {$x$},
                    ylabel = {$f(x)$},
                    legend style={at={(0.60,0.80)}, anchor=east},
                    ]

                    \addplot [domain=0:3.0, smooth, thick] { exp(x) };
                    
                    \addplot[color=blue, solid, smooth] table [x=x, y=y] {Ex2.txt};
                    \addplot[color=red, solid, smooth] table [x=x, y=y] {Ex4.txt};
                    
                    \addplot[color=black, mark=o, only marks, mark size=2.pt] coordinates {
                        (0,1)
                        (1,2.718281828)
                        (2,7.389056099)
                        (3,20.08553692)};
                    
                    \legend{$f(x)=e^x$, Spline Natural, Spline Fixada};
                \end{axis}
            \end{tikzpicture}
            \caption{Aproximação com splines da função exponencial: $f(x) = e^x$.}
            \label{fig:exp}
        \end{figure}
                
        \begin{figure}[hbt!] 
            \begin{tikzpicture}
                \begin{axis}[
                    grid=both,
                    xlabel = {$x$},
                    ylabel = {$\int_{0}^{x} e^x \,dx \; - \; \int_{0}^{x} S(x) \,dx$},
                    legend style={at={(0.65,0.20)}, anchor=east},
                    ]
                   
                    \addplot[color=blue, solid, smooth] table [x=x, y=eNatural] {Error_Ex2_4.txt};
                    \addplot[color=red, solid, smooth] table [x=x, y=eFixed] {Error_Ex2_4.txt};
                    
                    \addplot[color=black, mark=o, only marks, mark size=2.pt] coordinates {(0,0)(1,0)(2,0)(3,0)};
                    
                    \legend{Erro Spline Natural, Erro Spline Fixada};
                \end{axis}
            \end{tikzpicture}
            \caption{Erro da integral das aproximações com splines da função exponencial, entre 0 e $x$.}
            \label{fig:intexp}
        \end{figure}
        
        Para realizar o teste proposto foi utilizada uma das tabelas de VFP do modelo Unisim-II-H \cite{maschio2018case}. A tabela escolhida é de produção e tem os seus parâmetros descritos na Tabela \ref{table:tabvfp}.

        \begin{table*} 
            \caption{Parâmetros da tabela de VFP utilizada nas comparações. O parâmetro principal (BHP) é em $kgf/cm^2$.}
            \label{table:tabvfp}
            \begin{tabular}{ l l c c c c }
                \hline
                Variável & Descrição & Unidade & Valores & Mínimo & Máximo \\ 
                \hline
                LIQ  & Vazão de líquido         & $m^3/d$    & 6 & 200 &   3200 \\
                GLR  & Razão gás líquido        & $m^3/m^3$  & 8 & 30  &    240 \\
                WCUT & Corte de água            & $m^3/m^3$  & 6 &  0  &    0.9 \\
                LFG  & Vazão de \emph{gas-lift} & $m^3/d$    & 3 &  0  &  20$\,$000 \\
                WHP  & Pressão de cabeça        & $kgf/cm^2$ & 3 & 10  &     30 \\
                \hline
            \end{tabular}
        \end{table*}

        O teste realizado é o mesmo descrito em \cite{relatoriosplinesnaturais}. Para diferentes combinações dos parâmetros GLR, WCUT, LFG e WHP foram construídas funções interpoladoras dos pontos (LIQ, BHP). Em cada teste um dos pontos (LIQ, BHP) foi excluído. Este mesmo ponto foi posteriormente estimado com a função interpoladora.

        Para usar splines fixas é preciso fornecer a derivada da função a ser aproximada. A estimativa das derivadas nos extremos da função foi feita com os pares de valores conhecidos nos extremos (Figura \ref{fig:exvfp}):

        \begin{align*}
            f'(x_0) &\approx \frac{y_1-y_0}{x_1-x_0} \beta_0 \\
            f'(x_n) &\approx \frac{y_n-y_{n-1}}{x_n-x_{n-1}} \beta_n \\
        \end{align*}

        \begin{figure}[hbt!] 
            \begin{tikzpicture}
                \begin{axis}[
                    grid=both,
                    xlabel = {$Q \; (m^3/d)$},
                    ylabel = {$P_{wf} \; (kgf/cm^2)$},
                    legend style={at={(0.40,0.90)}, anchor=east},
                    ]
                   
                    \addplot[color=red, solid, thick] coordinates {(100,413.4655885)(600,406.039331)};
                    \addplot[color=blue, solid, thick] coordinates {(2000,421.797408)(3400,447.4062115)};
                    \addplot[color=black, solid, smooth, thick] table [x=x, y=y] {vfp/P1_Ok.txt};
                    \addplot[color=black, mark=o, only marks, mark size=2.pt] table [x=x, y=y] {vfp/P1_Ok_True.txt};
                    
                    \legend{$\approx f'(x_0)$, $\approx f'(x_n)$};
                \end{axis}
            \end{tikzpicture}
            \caption{Aproximação das derivadas nos extremos do domínio.}
            \label{fig:exvfp}
        \end{figure}
        
        Como as derivadas foram estimadas utilizando os dois primeiros e os dois últimos pontos de (LIQ, BHP), avaliou-se que não faria sentido avaliar pontos fora do domínio inicial. Desta forma, foram realizados 4 testes para cada combinação dos demais parâmetros, excluindo apenas pontos internos. 

        Os parâmetros $\beta$ foram utilizados para buscar uma melhor estimativa destas derivadas. Foram feitas sensibilidades do erro médio e do desvio padrão do erro em função de $\beta_0$ e $\beta_n$.
        
        As Figuras \ref{fig:beta0media} e \ref{fig:beta0devpad} mostram que, como esperado, as estimativas mais afetadas pelo valor de $\beta_0$ são os pontos LIQ=400 e 800. Existe um tendência de redução do erro médio com valores baixos de $\beta_0$. Já o gráfico do desvio padrão dos erros mostra que existe uma região \emph{ótima} aproximadamente entre 1.4 e 2.4. Não existe um valor único que minimiza o desvio padrão dos erros de cada valor de LIQ. Optou-se por usar um valor intermediário: $\beta_0=1.8$. Com este valor os desvios padrão de LIQ=400 e 800 ficam aproximadamente iguais ou melhores que os valores com interpolação linear ou com splines naturais.

        \begin{figure}[hbt!] 
            \begin{tikzpicture}
                \begin{axis}[
                    grid=both,
                    xlabel = {$\beta_0$},
                    ylabel = {Erro médio $(kgf/cm^2)$},
                    legend style={at={(0.40,0.99)}, anchor=east},
                    ]
                    \addplot[color=black, solid, smooth, thick] table [x=Beta0, y=Mean400] {vfp/P1_Beta0.txt};
                    \addplot[color=red, solid, smooth, thick] table [x=Beta0, y=Mean800] {vfp/P1_Beta0.txt};
                    \addplot[color=blue, solid, smooth, thick] table [x=Beta0, y=Mean1600] {vfp/P1_Beta0.txt};
                    \addplot[color=green, solid, smooth, thick] table [x=Beta0, y=Mean2400] {vfp/P1_Beta0.txt};
                    
                    \addplot[color=black, densely dotted, smooth, thick] table [x=Beta0, y=MeanLin400] {vfp/P1_Beta0.txt};
                    \addplot[color=black, dashed, smooth, thick] table [x=Beta0, y=MeanSplNat400] {vfp/P1_Beta0.txt};

                    \addplot[color=red, densely dotted, smooth, thick] table [x=Beta0, y=MeanLin800] {vfp/P1_Beta0.txt};
                    \addplot[color=red, dashed, smooth, thick] table [x=Beta0, y=MeanSplNat800] {vfp/P1_Beta0.txt};
                    
                    \legend{LIQ=400, LIQ=800, LIQ=1600, LIQ=2400};
                \end{axis}
            \end{tikzpicture}
            \caption{Média dos erros de BHP com splines fixas em função do valor de $\beta_0$. Resultados da interpolação linear em pontilhado e splines naturais em tracejado.}
            \label{fig:beta0media}
        \end{figure}

        \begin{figure}[hbt!] 
            \begin{tikzpicture}
                \begin{axis}[
                    grid=both,
                    xlabel = {$\beta_0$},
                    ylabel = {Desvio Padrão do Erro $(kgf/cm^2)$},
                    legend style={at={(0.40,0.99)}, anchor=east},
                    ]
                    \addplot[color=black, solid, smooth, thick] table [x=Beta0, y=StdErr400] {vfp/P1_Beta0.txt};
                    \addplot[color=red, solid, smooth, thick] table [x=Beta0, y=StdErr800] {vfp/P1_Beta0.txt};
                    \addplot[color=blue, solid, smooth, thick] table [x=Beta0, y=StdErr1600] {vfp/P1_Beta0.txt};
                    \addplot[color=green, solid, smooth, thick] table [x=Beta0, y=StdErr2400] {vfp/P1_Beta0.txt};
                    
                    \addplot[color=black, densely dotted, smooth, thick] table [x=Beta0, y=StdErrLin400] {vfp/P1_Beta0.txt};
                    \addplot[color=black, dashed, smooth, thick] table [x=Beta0, y=StdErrSplNat400] {vfp/P1_Beta0.txt};

                    \addplot[color=red, densely dotted, smooth, thick] table [x=Beta0, y=StdErrLin800] {vfp/P1_Beta0.txt};
                    \addplot[color=red, dashed, smooth, thick] table [x=Beta0, y=StdErrSplNat800] {vfp/P1_Beta0.txt};
                    
                    \legend{LIQ=400, LIQ=800, LIQ=1600, LIQ=2400};
                \end{axis}
            \end{tikzpicture}
            \caption{Desvio padrão dos erros de BHP com splines fixas em função do valor de $\beta_0$. Resultados da interpolação linear em pontilhado e splines naturais em tracejado.}
            \label{fig:beta0devpad}
        \end{figure}

        O parâmetro $\beta_n$ tem impacto significativo apenas nos valores de LIQ=1600 e 2400 (Figura \ref{fig:betanmedia}). O valor ótimo para a média do erro de LIQ=2400 é aproximadamente 0.9, enquanto que para LIQ=1600 o ótimo é em valores menores de $\beta_n$. A Figura \ref{fig:betandevpad} mostra que apenas o desvio padrão de LIQ=2400 é sensível ao valor de $\beta_n$. Foi utilizado $\beta_n=0.9$. Com este valor o erro médio se aproxima de zero, e o desvio padrão do erro de LIQ=2400 é mínimo, ficando menor que o da interpolação linear ou com splines naturais.
        
        \begin{figure}[hbt!] 
            \begin{tikzpicture}
                \begin{axis}[
                    grid=both,
                    xlabel = {$\beta_n$},
                    ylabel = {Erro médio $(kgf/cm^2)$},
                    legend style={at={(0.40,0.99)}, anchor=east},
                    ]
                    \addplot[color=black, solid, smooth, thick] table [x=Betan, y=Mean400] {vfp/P1_Betan.txt};
                    \addplot[color=red, solid, smooth, thick] table [x=Betan, y=Mean800] {vfp/P1_Betan.txt};
                    \addplot[color=blue, solid, smooth, thick] table [x=Betan, y=Mean1600] {vfp/P1_Betan.txt};
                    \addplot[color=green, solid, smooth, thick] table [x=Betan, y=Mean2400] {vfp/P1_Betan.txt};
                    
                    \addplot[color=blue, densely dotted, smooth, thick] table [x=Betan, y=MeanLin1600] {vfp/P1_Betan.txt};
                    \addplot[color=blue, dashed, smooth, thick] table [x=Betan, y=MeanSplNat1600] {vfp/P1_Betan.txt};

                    \addplot[color=green, densely dotted, smooth, thick] table [x=Betan, y=MeanLin2400] {vfp/P1_Betan.txt};
                    \addplot[color=green, dashed, smooth, thick] table [x=Betan, y=MeanSplNat2400] {vfp/P1_Betan.txt};
                    
                    \legend{LIQ=400, LIQ=800, LIQ=1600, LIQ=2400};
                \end{axis}
            \end{tikzpicture}
            \caption{Média dos erros de BHP com splines fixas em função do valor de $\beta_n$. Resultados da interpolação linear em pontilhado e splines naturais em tracejado.}
            \label{fig:betanmedia}
        \end{figure}
        
        \begin{figure}[hbt!] 
            \begin{tikzpicture}
                \begin{axis}[
                    grid=both,
                    xlabel = {$\beta_n$},
                    ylabel = {Desvio Padrão do Erro $(kgf/cm^2)$},
                    legend style={at={(0.40,0.99)}, anchor=east},
                    ]
                    \addplot[color=black, solid, smooth, thick] table [x=Betan, y=StdErr400] {vfp/P1_Betan.txt};
                    \addplot[color=red, solid, smooth, thick] table [x=Betan, y=StdErr800] {vfp/P1_Betan.txt};
                    \addplot[color=blue, solid, smooth, thick] table [x=Betan, y=StdErr1600] {vfp/P1_Betan.txt};
                    \addplot[color=green, solid, smooth, thick] table [x=Betan, y=StdErr2400] {vfp/P1_Betan.txt};
                    
                    \addplot[color=blue, densely dotted, smooth, thick] table [x=Betan, y=StdErrLin1600] {vfp/P1_Betan.txt};
                    \addplot[color=blue, dashed, smooth, thick] table [x=Betan, y=StdErrSplNat1600] {vfp/P1_Betan.txt};

                    \addplot[color=green, densely dotted, smooth, thick] table [x=Betan, y=StdErrLin2400] {vfp/P1_Betan.txt};
                    \addplot[color=green, dashed, smooth, thick] table [x=Betan, y=StdErrSplNat2400] {vfp/P1_Betan.txt};
                    
                    \legend{LIQ=400, LIQ=800, LIQ=1600, LIQ=2400};
                \end{axis}
            \end{tikzpicture}
            \caption{Desvio padrão dos erros de BHP com splines fixas em função do valor de $\beta_n$. Resultados da interpolação linear em pontilhado e splines naturais em tracejado.}
            \label{fig:betandevpad}
        \end{figure}
        
        
        As Figuras \ref{fig:errosvfpspline}, \ref{fig:errosvfplinear} e \ref{fig:errosvfpsplinefixa} mostram que as splines fixas com $\beta_0$ e $\beta_n$ \emph{otimizados}\footnote{$\beta_0=1.8$ e $\beta_n=0.9$.} alcançaram, em geral, resultados melhores que as splines naturais ou interpolação linear. Em comparação com a interpolação linear, o ganho foi mais significativo em LIQ=400, que é a região em que a variação da derivada da função é mais forte. Nesta região a imposição da derivada teve impacto positivo. Por outro lado, em LIQ=800 a spline fixa apresentou resultados ruins, com o erro médio \emph{longe} do zero.  
        
        \begin{figure}[hbt!] 
            \begin{tikzpicture}
                \begin{axis}
                    [
                    ytick={1,2,3,4,5,6},
                    yticklabels={LIQ=400, LIQ=800, LIQ=1600, LIQ=2400},
                    xmin=-60,
                    xmax=100,
                    ]
                    \addplot+[
                    boxplot prepared={
                        median=1.098113864,
                        upper quartile=5.97379165075,
                        lower quartile=-20.46857672,
                        upper whisker=10.27655936,
                        lower whisker=-51.3246126,
                    }, fill=gray, draw=black, solid
                    ] coordinates {};
                    \addplot+[
                    boxplot prepared={
                        median=0.60786930725,
                        upper quartile=27.1931668275,
                        lower quartile=-5.54907854075,
                        upper whisker=68.1779129,
                        lower whisker=-11.96436218,
                    }, fill=gray, draw=black, solid
                    ] coordinates {};
                    \addplot+[
                    boxplot prepared={
                        median=-3.5615469125,
                        upper quartile=-0.41523451875,
                        lower quartile=-10.7981950575,
                        upper whisker=4.21248588,
                        lower whisker=-24.41612724,
                    }, fill=gray, draw=black, solid
                    ] coordinates {};
                    \addplot+[
                        boxplot prepared={
                        median=1.203924668,
                        upper quartile=3.92193537925,
                        lower quartile=-0.4740216875,
                        upper whisker=10.33980045,
                        lower whisker=-2.678721812,
                    }, fill=gray, draw=black, solid
                    ] coordinates {};
                \end{axis}
            \end{tikzpicture}
        \caption{Erros na estimativa de BHP com splines naturais, em função do valor de LIQ excluído.}
        \label{fig:errosvfpspline}
        \end{figure}

        \begin{figure}[hbt!] 
            \begin{tikzpicture}
                \begin{axis}
                    [
                    ytick={1,2,3,4,5,6},
                    yticklabels={LIQ=400, LIQ=800, LIQ=1600, LIQ=2400},
                    xmin=-60,
                    xmax=100,
                    ]
                    \addplot+[
                    boxplot prepared={
                        median=2.3675538335,
                        upper quartile=8.39584525025,
                        lower quartile=-23.50684658,
                        upper whisker=13.38849467,
                        lower whisker=-58.92813933,
                    }, fill=gray, draw=black, solid
                    ] coordinates {};
                    \addplot+[
                    boxplot prepared={
                        median=8.226421333,
                        upper quartile=11.3735147525,
                        lower quartile=-4.374550333,
                        upper whisker=21.19345867,
                        lower whisker=-11.23103167,
                    }, fill=gray, draw=black, solid
                    ] coordinates {};
                    \addplot+[
                    boxplot prepared={
                        median=3.76609575,
                        upper quartile=7.743125625,
                        lower quartile=-2.642448,
                        upper whisker=16.4773995,
                        lower whisker=-9.746704,
                    }, fill=gray, draw=black, solid
                    ] coordinates {};
                    \addplot+[
                        boxplot prepared={
                        median=1.56718725,
                        upper quartile=3.031477,
                        lower quartile=-1.094461875,
                        upper whisker=6.603298,
                        lower whisker=-4.424486,
                    }, fill=gray, draw=black, solid
                    ] coordinates {};
                \end{axis}
            \end{tikzpicture}
        \caption{Erros na estimativa de BHP com interpolação linear, em função do valor de LIQ excluído.}
        \label{fig:errosvfplinear}
        \end{figure}

        \begin{figure}[hbt!] 
            \begin{tikzpicture}
                \begin{axis}
                    [
                    ytick={1,2,3,4},
                    yticklabels={LIQ=400, LIQ=800, LIQ=1600, LIQ=2400},
                    xmin=-60,
                    xmax=100,
                    ]
                    \addplot+[
                    boxplot prepared={
                        median=-9.5158550825,
                        upper quartile=3.602819047,
                        lower quartile=-12.4230881375,
                        upper whisker=10.53613859,
                        lower whisker=-16.64090699,
                    }, fill=gray, draw=black, solid
                    ] coordinates {};
                    \addplot+[
                    boxplot prepared={
                        median=9.8030380835,
                        upper quartile=19.73523468,
                        lower quartile=6.3020903365,
                        upper whisker=39.88495119525,
                        lower whisker=1.586584701,
                    }, fill=gray, draw=black, solid
                    ] coordinates {};
                    \addplot+[
                    boxplot prepared={
                        median=-6.2403972465,
                        upper quartile=-3.74610976575,
                        lower quartile=-7.98734292575,
                        upper whisker=-0.4385887618,
                        lower whisker=-13.70546997,
                    }, fill=gray, draw=black, solid
                    ] coordinates {};
                    \addplot+[
                    boxplot prepared={
                        median=0.2757144176,
                        upper quartile=0.896100179675,
                        lower quartile=-0.387540364275,
                        upper whisker=2.632459095,
                        lower whisker=-2.3130011802,
                    }, fill=gray, draw=black, solid
                    ] coordinates {};
                \end{axis}
            \end{tikzpicture}
        \caption{Erros na estimativa de BHP com splines fixas (com $\beta_0=1.8$ e $\beta_n=0.9$), em função do valor de LIQ excluído.}
        \label{fig:errosvfpsplinefixa}
        \end{figure}

        O código foi implementado em C e em um único arquivo. Pode ser encontrado em \href{https://github.com/TiagoCAAmorim/numerical-methods/blob/main/04_Splines/04_splines.c}{https://github.com/Tiago CAAmorim/numerical-methods}.

    \section{Conclusão}
    
        O desempenho de splines fixas para interpolar valores de pressão de fundo a partir de uma tabela de VFP foi, em geral, melhor que o da interpolação linear. A imposição de aproximações das derivadas nos extremos do domínio teve impacto positivo na redução da variabilidade dos erros. A diferença para a interpolação linear não parece ser significativa a ponto de justificar a utilização de um método de interpolação muito mais custoso computacionalmente que a interpolação linear.

    % \label{}
    
%% The Appendices part is started with the command \appendix;
%% appendix sections are then done as normal sections

\appendix

% \section{Funções de Teste}

%% \section{}
%% \label{}

%% If you have bibdatabase file and want bibtex to generate the
%% bibitems, please use
%%

\bibliographystyle{elsarticle-num} 
\bibliography{refs}

%% else use the following coding to input the bibitems directly in the
%% TeX file.

% \begin{thebibliography}{00}

%% \bibitem{label}
%% Text of bibliographic item

% \bibitem{}

% \end{thebibliography}

% \newpage
% \FloatBarrier
% \section{Código em C}

% O código de ambos métodos foi implementado em um único arquivo. O código é apresentado em duas partes neste documento para facilitar a leitura. O código pode ser encontrado em \href{https://github.com/TiagoCAAmorim/numerical-methods}{https://github.com/TiagoCAAmorim/numerical-methods}.

% \subsection{Método da Bissecção}
% \lstinputlisting[language=C, linerange={1-229}]{./02_newton_raphson.c}

% \subsection{Método de Newton-Raphson}
% \lstinputlisting[language=C, linerange={231-445}]{./02_newton_raphson.c}

% \subsection{Método da Mínima Curvatura}
% \lstinputlisting[language=C, linerange={448-958}]{./02_newton_raphson.c}

\end{document}
\endinput